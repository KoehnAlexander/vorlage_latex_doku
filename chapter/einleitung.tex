\chapter{Einleitung}
\label{cha:Einleitung}
Technische Bewertung eines Fahrzeugprototypen und aufbauender Konzeptentwurf für ein digitales Ökosystem für Fahrzeuge.

Das Ziel dieser Arbeit ist basierend auf einer Bewertung eines bestehenden Fahrzeugprototypen einen Entwurf für ein mögliches digitales Ökosystem zur Steigerung des Kundenerlebnisses zu schaffen. 


Das Ziel dieser Arbeit ist es zu erläutern und Lösungen aufzuzeigen, wie digitale Technologien in Fahrzeugen implementiert werden können, um das Erlebnis für die Fahrzeugbesitzer und Betrachter im Rahmen eines digitalen Ökosystems zu steigern.
Beispiele für verwendete digitalen Technologien in dieser Arbeit sind Non-Fungible Token (NFT\nomenclature{NFT}{Non-Fungible Token}), Virtual Reality (VR\nomenclature{VR}{Virtual Reality}) und Künstliche Intelligenz (KI\nomenclature{KI}{Künstliche Intelligenz}).
Ein bestehender Fahrzeugprototyp mit erweiterten Individualisierungsmöglichkeiten wurde genutzt, um Möglichkeiten für technische Änderungen im Fahrzeug zu zeigen.

Die Arbeit ist wie folgt gegliedert:\\
Zuerst werden in Kapitel \ref{cha:Grundlagen} Grundlagen zu unterschiedlichen für diese Arbeit wichtige Technologien erläutert und der Fahrzeugprototyp in Kapitel \ref{cha:Prototyp} näher beschrieben. In Kapitel \ref{cha:Kriterien} werden Kriterien für den Einzug von Individualisierungsmöglichkeiten mit digitalen Technologien im Fahrzeug aufgezählt und anschließend in Kapitel \ref{cha:Bewertung} der Prototyp anhand dieser bewertet. Mithilfe der Bewertung bildet Kapitel \ref{cha:Konzeptentwurf} einen möglichen Entwurf für das Einbinden der digitalen Technologien. Auf Basis des Konzeptentwurfs wird in Kapitel \ref{cha:Verifikation} darüber unter unterschiedlichen Gesichtspunkten diskutiert. Abschließend wird in Kapitel \ref{cha:zusammenfassung} die Arbeit auf wesentliche Erläuterungen zusammengefasst.

\chapter{Grundlagen}
\label{cha:Grundlagen}
Im folgenden werden für diese Arbeit notwendige Grundlagen erläutert. Die Reihenfolge der Erklärungen sortiert sich von allgemeinen Grundlagen zu digitalen Technologien über verbaute Technologien zu fahrzeugspezifischen Kenntnissen.
\section{Digitale Technologien}
In dieser Arbeit genutzte digitalen Technologien sind zum einen Blockchain Technologien, worauf NFT basieren, und in diesem Kontext die Veränderungen durch Web3. Daneben werden Virtual Reality (VR\nomenclature{VR}{Virtual Reality})und Augmented Reality (AR\nomenclature{AR}{Augmented Reality}) kurz erläutert, da diese im weiteren Verlauf der Arbeit genutzt werden.
\subsection{Blockchain}

\subsection{Non-Fungible Token}
\subsection{Web3}
\subsection{Virtual Reality}
\subsection{Augmented Reality}
\section{Technologien}
\subsection{LED}
\subsection{Matrix LED}
\subsection{Displays}
\subsection{Projektoren}
\subsection{E Ink Folien}
\subsection{Morphende Oberflächen}
\section{Fahrzeugtechnik}
\subsection{Fahrzeugentwicklung}
\subsection{Elektrik/Elektronik Architektur}
\subsection{Bussysteme}
\subsection{Rechtliche Rahmenbedingungen}
\subsection{Sicherheitsbedingungen}


\chapter{Prototyp}
\label{cha:Prototyp}
\subsection{Beschreibung}
\subsection{Vision}
\subsection{Komponenenten}
\subsection{Implementierung}


\chapter{Kriterien für den Einzug von Individualisierungsmöglichkeiten mit digitalen Technologien im Fahrzeug }
\label{cha:Kriterien}

... Text Konzeptentwurf: Gegenüberstellung verschiedener Lösungsansätze und Lösungsgenerierung, etc.

\chapter{Bewertung des Prototypen}
\label{cha:Bewertung}

\chapter{Konzeptentwurf}
\label{cha:Konzeptentwurf}

... Text Umsetzung: Beschreibung der Umsetzung und eigener Untersuchungen ...



\chapter{Verifikation und Diskussion}
\label{cha:Verifikation}

... Verifikation, Auswertung, Lösungsbewertung, Diskussion der Ergebnisse

\chapter{Zusammenfassung}
\label{cha:zusammenfassung}

... Text Zusammenfassung und Ausblick: In der Zusammenfassung unbedingt klare Aussagen zum Ergebnis der Arbeit nennen, im Optimalfall quantitative Angaben. Die Inhalte müssen sich auf die Fragestellung aus der Einleitung  beziehen. ...