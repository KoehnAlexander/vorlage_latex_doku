\chapter{Einleitung}
\label{cha:Einleitung}
Das Ziel dieser Arbeit ist basierend auf einer Technischen Bewertung eines Fahrzeugprototypen Möglichkeiten aufzuzeigen wie die dort verwendeten Technologien und Komponenten in Serienfahrzeuge integriert werden können.

Beispiele für verwendete digitalen Technologien in dieser Arbeit sind Non-Fungible Token (NFT\nomenclature{NFT}{Non-Fungible Token}), Virtual Reality (VR\nomenclature{VR}{Virtual Reality}) und Künstliche Intelligenz (KI\nomenclature{KI}{Künstliche Intelligenz}).
Ein bestehender Fahrzeugprototyp mit erweiterten Individualisierungsmöglichkeiten wurde genutzt, um Möglichkeiten für technische Änderungen im Fahrzeug zu zeigen.

Die Arbeit ist wie folgt gegliedert:\\
Zuerst werden in Kapitel \ref{cha:Grundlagen} Grundlagen zu unterschiedlichen für diese Arbeit wichtige Technologien erläutert und der Fahrzeugprototyp in Kapitel \ref{cha:Prototyp} näher beschrieben. In Kapitel \ref{cha:Kriterien} werden Kriterien für den Einzug von Individualisierungsmöglichkeiten durch neuartige Komponenten im Fahrzeug aufgezählt und anschließend in Kapitel \ref{cha:Bewertung} der Prototyp anhand dieser bewertet. Mithilfe der Bewertung bildet Kapitel \ref{cha:Konzeptentwurf} einen möglichen Entwurf für das Einbinden der digitalen Technologien. Auf Basis des Konzeptentwurfs wird in Kapitel \ref{cha:Verifikation} darüber unter unterschiedlichen Gesichtspunkten diskutiert. Abschließend wird in Kapitel \ref{cha:zusammenfassung} die Arbeit auf wesentliche Erläuterungen zusammengefasst.