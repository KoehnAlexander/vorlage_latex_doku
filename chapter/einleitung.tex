\chapter{Einleitung}
\label{cha:Einleitung}
Das Ziel dieser Arbeit ist basierend auf einer Analyse von Komponenten eines Fahrzeugprototypen unter unterschiedlichen Kriterien Anforderungen  zu stellen, welche Entwicklungen benötigt werden, damit diese Komponenten in eine Fahrzeugserie eingebaut werden.\\
Mit Hilfe dieser Arbeit wird eine Diskussionsgrundlage für die weitere Verwendung der Technologien des Prototypen geschaffen, um das Gesamtkonzept des Prototypen in die Fahrzeugserienentwicklung zu integrieren. \\
Der Prototyp ist ein Forschungsfahrzeug, das durch eingebaute Komponenten im Interieur und Exterieur sein Erscheinungsbild für Beobachter und Beobachterinnen dynamisch verändern kann. Dynamisch bedeutet in diesem Fall, dass das Fahrzeug erstens dynamische Effekte besitzt und zweitens diese Effekte auf andere Erscheinungsbilder umschaltbar sind.
Die Arbeit ist wie folgt gegliedert:\\
Zuerst werden in Kapitel \ref{cha:Grundlagen} Grundlagen zu den wichtigsten in dieser Arbeit behandelten Technologien vermittelt, um auf diesen Grundlagen den Fahrzeugprototyp mit seinen Komponenten in Kapitel \ref{cha:Prototyp} näher beschreiben zu können. Daneben wird in dem Kapitel noch näher auf das Grundkonzept des Prototypen eingegangen.\\
Aufbauend auf den Erläuterungen zu den einzelnen Komponenten des Fahrzeuges werden diese im Kapitel \ref{cha:Analyse} durch zuerst definierte Kriterien überprüft, um mit dieser Analyse anschließend Anforderungen an die weitere Entwicklung der Komponenten in Kapitel \ref{cha:Konzeptentwurf} zu stellen.\\
In Kapitel \ref{cha:Verifikation} werden diese Anforderungen unter dem gesamten Leitbild betrachtet und ein Ausblick auf mögliche zukünftige Szenarien gestellt. Abschließend wird in Kapitel \ref{cha:zusammenfassung} die Arbeit auf die wesentlichen Erkenntnisse zusammengefasst.