\chapter{Einleitung}
\label{cha:Einleitung}
Das Ziel dieser Arbeit ist, basierend auf einer Analyse von technischen Komponenten eines Fahrzeugprototypen, unter unterschiedlichen Kriterien Anforderungen zu stellen, welche Entwicklungen an den definierten Komponenten und Fahrzeugen für eine Serienreife benötigt werden. Serienreife bedeutet in dieser Arbeit, dass die nötigen Bedingungen für eine Vorentwicklung und Serienentwicklung erfüllt sind.\\
Mit Hilfe dieser Arbeit wird eine Diskussionsgrundlage für die weitere Verwendung der Technologien und Komponenten des Prototypen geschaffen, um das Gesamtkonzept des Prototypen in die Fahrzeugserienentwicklung zu integrieren.\\
Der Prototyp ist ein Forschungsfahrzeug, das durch eingebaute Komponenten im Interieur und Exterieur sein Erscheinungsbild für Beobachter und Beobachterinnen dynamisch verändern kann. Dynamisch bedeutet in diesem Fall, dass das Fahrzeug erstens dynamische visuelle und akustische Effekte besitzt und zweitens diese Effekte auf andere Erscheinungsbilder umschaltbar sind.
Die Arbeit ist wie folgt gegliedert:\\
Zuerst werden im Kapitel \ref{cha:Grundlagen} Grundlagen zu den wichtigsten in dieser Arbeit behandelten Technologien vermittelt, um auf Diesen den Fahrzeugprototyp mit seinen Komponenten im Kapitel \ref{cha:Prototyp} näher beschreiben zu können. Daneben wird in diesem Kapitel noch näher auf die Vision und das Grundkonzept des Prototypen eingegangen, um den Sinn der Komponenten zu erklären.\\
Aufbauend auf den Erläuterungen zu den einzelnen Komponenten des Fahrzeuges werden diese im Kapitel \ref{cha:Analyse} durch zuerst definierte Kriterien überprüft, um mit dieser Analyse anschließend Anforderungen an die weitere Entwicklung der Komponenten und an das Fahrzeug im Kapitel \ref{cha:Konzeptentwurf} zu stellen.\\
Im Kapitel \ref{cha:Verifikation} werden diese Anforderungen unter dem gesamten Leitbild betrachtet und ein Ausblick auf mögliche zukünftige Szenarien vorgestellt. Abschließend wird im Kapitel \ref{cha:zusammenfassung} die Arbeit auf die wesentlichen Erkenntnisse zusammengefasst.