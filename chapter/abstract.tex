\chapter*{Kurzfassung} %*-Variante sorgt dafür, das Abstract nicht im Inhaltsverzeichnis auftaucht
Die folgende Arbeit untersucht die Komponenten eines Fahrzeugprototypen auf Basis von definierten Kriterien im Bereich rechtlich, wirtschaftlich und technisch. Das untersuchte Fahrzeug besitzt für die Erzeugung einer künstlerischen Gesamtinszenierung unterschiedliche digitale Sonderausstattungen, wie zum Beispiel Bildschirme, Videoprojektoren und Lichterzeuger. \\
Die Untersuchungsergebnisse wurden durch eigenständige Recherche und in Absprache mit Fachleuten ermittelt. Das Ziel der Untersuchung ist zum einen erste Erkenntnisse für die Weiterentwicklung der Komponenten zu erhalten. Zum anderen ist das Ziel mit den Analyseergebnissen Konzepte für die Ansteuerung der Komponenten an Fahrzeugbusarchitekturen zu entwickeln und zu bewerten. \\
In der Analyse wurde festgestellt, dass die Komponenten ein leistungsstarkes Bussystem benötigen, wie zum Beispiel MOST (Media Oriented Systems Transport\nomenclature{MOST}{Media Oriented Systems Transport}) oder Automotive Ethernet, um einen möglichst großen Vorteil für die Kunden zu erzielen. Der Konzeptentwurf mit einer dezentralen Speicherung der Daten ist im Vergleich die effektivste Architektur, da dort geringere Nutzdatenraten gefordert sind.
\cleardoublepage