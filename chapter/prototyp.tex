\chapter{Fahrzeugprototyp}
\label{cha:Prototyp}
\section{Gesamtkonzept}
Das Gesamtkonzept für den Fahrzeugprototypen basiert auf der Vision eines Fahrzeuges als Leinwand (englisch: car as a canvas). Das Zielbild dieser Vision ist ein Fahrzeug, das auf vollständiger Weise seine Wahrnehmung auf den Menschen verändern kann. Besonders die optische Wahrnehmung ist hier im Vordergrund der Vision, aber diese bleibt nicht singulär, sondern die anderen Wahrnehmungsarten sind genauso veränderbar.\\

Unter der übergeordneten Vision das Fahrzeug als Leinwand zu betrachten, bildet das Gesamtkonzept einen mögliche Variante der Vision. 
Durch unterschiedliche Komponenten werden die optischen, haptischen, olfaktorischen und akustischen Individualisierungsmöglichkeiten vergrößert. Diese Komponenten können im Gegensatz zu vielen bisherigen Techniken dynamisch ihre Inhalte verändern und somit das Erscheinungsbild des Fahrzeuges über die Zeit verändern. Die Inhalte aller Kollektion sind abgestimmt unter einer Kollektion für den Besitzer auswählbar und mit anderen Kollektionen austauschbar.
\section{Beschreibung}
Zugrundeliegend für das Gesamtkonzept, was unten näher beschrieben wird, die Vision eines Fahrzeuges als Leinwand. Es soll für den Kunden möglich gemacht werden ihr Fahrzeug in einer neuen Art individualisieren zu können. Die Individualisierungen sollen durch neuartige eingebaute Komponenten und durch Augmented Reality Inhalte implementiert werden. Auf die Augmented Reality Inhalte wird in der folgenden Arbeit nicht näher eingegangen.\\
Der Prototyp basiert auf einem elektrischen Mittelklasse Serienfahrzeug gebaut im Jahr 2020. Aufbauend auf diesem Fahrzeug wurden im Exterieur und Interieur Teile ergänzt und teilweise mit anderen Komponenten ausgetauscht.

\section{Komponenten}
Das Fahrzeug hat sowohl im Exterieur als auch im Interieur Komponenten verbaut. Die Komponenten wurden nach der verwendeten Technik und dem Ort benannt. Die Einteilung erfolgt nach der Betrachtungsweise innerhalb oder außerhalb des Fahrzeugs der Komponenten. Exterieur Komponenten werden von Beobachtern außerhalb des Fahrzeugs betrachtet. Interieur Komponenten entsprechend von innen.\\
Im Exterieur sind dies ein E-Ink Display im Frontkühlergrill, ein durchgehendes RGB-Leuchtband in der Frontschürze, E-Ink Embleme über den beiden vorderen Radkästen, RGB-Leuchtbänder in allen vier Radkästen, Beamer in den beiden Außenspiegel, nach außen gerichtete Displays in den Fondtürfenstern, ein RGB Leuchtband in der Heckleuchte und zwei kleine E-Ink Displays unterhalb der Heckleuchte.
Im Interieur sind dies ein durchgehendes RGB LED Leuchtband von den hinteren Türen über die vorderen Türen bis über das gesamte Cockpit, in den Türen ein LED Matrix Feld, Displays in der Einstiegsleiste der vorderen Türen, Beamer in den Fußraum der Frontsitze, andere Designs für das Fahrer und das Zentraldisplay, eine morphende Oberfläche in der Mittelkonsole, ein durchsichtiges LCD Display für das Dachfenster und eine LED Matrix für den hinteren Teil des Dachhimmels.
Im folgenden werden alle Komponenten nähe betrachtet.
\subsection{E-Ink Display im Frontkühlergrill}
Das E-Ink Display befindet sich hinter einer Scheibe mit einem Markenlogo in der Mitte der Fahrzeugfront und schließt an den Seiten auf die beiden Frontlichter an. Das Display kann statische Bilder zeigen und soll mit einem großen Betrachtungswinkel von vorne gesehen werden.
\subsection{RGB-Leuchtband in der Frontschürze}
Das RGB-Leuchtband ist dreiteilig aufgeteilt. Zweit Teile befinden sich in der Frontleuchten und schließen auf gleicher Höhe mit dem Mittelstück an. Das Mittelstück befindet sich oberhalb des E-Ink Displays im Frontkühlergrill. Das Leuchtband kann dynamische Lichtinszenierungen erzeugen.
\subsection{E-Ink Embleme über den vorderen Radkästen}
Oberhalb der Radkästen befinden sich ca 20 cm Breite und 8 cm hohe E-Ink Displays. Diese können statisch Bilder darstellen und werden zum Anzeigen des Namens der verwendeten Kollektion genutzt.
\subsection{Leuchtstreifen in den Radkästen}
In allen vier Radkästen befinden sich RGB LED-Streifen am äußeren Rand und strahlen im Radkasten Innenraum auf den oberen Halbkreis des Reifenprofils. Die Beleuchtung ist dynamisch ansteuerbar.
\subsection{Beamer in den Außenspiegeln}
In den Außenspiegeln wurde der Innenraum mit der Spiegelmechanik ausgebaut und Beamer eingebaut. Der nach unten ausgerichtete Beamer bestrahlt die Flächen vor den vorderen Türen mit farbigen Videos.
\subsection{Displays in den Fonttürfenstern}
In den hinteren Türen befinden sich hinter der Nebenscheiben, die mit einer Leiste von den beweglichen Hauptglasscheiben getrennt sind, farbige LCD Dislpays. Diese können von außen betrachtet werden. Dir Rückseite der Displays ist von innen mit einer schwarzen Kunststoffverkleidung für die Passagiere abgedeckt.
\subsection{Leuchtband in der Heckleuchte}
In der Serienheckleuchte wurde das rote Leuchtband mit einem RGB LED-Streifen getauscht, um alle Farben darzustellen. Der Streifen ist dynamisch ansteuerbar.
\subsection{E-Ink Displays in der Heckleuchte}
Direkt unterhalb der Heckleuchten sind zwei E-Ink Displays eingebaut, um den Namen der Designs wie beim Display oberhalb des Radkastens für Betrachter von hinten zu zeigen.
\subsection{Interieur Leuchtband}
Der RGB LED-Streifen ist fünfgeteilt und erstreckt sich im oberen Bereich der Türverkleidung und schließt über das Cockpit zu einem einheitlichen Band ab. Der Streifen ist mit einer Streulichtabdeckung versehen, damit der Betrachter nicht die einzelnen LED erkennen kann. Der Streifen kann dynamisch Farben und Inszenierungen abspielen.
\subsection{Matrix LED Türtafeln}
In allen vier Türverkleidungen befinden sich unterhalb des LED-Streifens ein LED Feld hinter einer Abdeckung mit durchsichtigen Sternen. Die Sterne können somit mit unterschiedlichen Farben angestrahlt werden. 
\subsection{Displays in der Einstiegsleiste}
Anstelle einer Edelstahl Abdeckung mit einem Schriftzug befinden sich in den vorderen Türen Displays in der Einstiegsleiste. Die LCD Displays können bei geöffneter Türe Inhalte dem Betrachter darstellen.
\subsection{Beamer im Fußraum}
Für die vorderen Fußräume wurden zwei Beamer verbaut. Der eine Beamer befindet sich unterhalb der Lenkersäule, der Andere unterhalb des Handschuhfachs. Beide Beamer strahlen den Fußraum an und können dynamisch Inhalte abspielen.
\subsection{Morphende Oberläche in der Mittelkonsole}
In der Mittelkonsole wurde das Ablagefach und die abgelederte Abdeckung durch eine neuartige Vorrichtung ersetzt, die von innen mit Hilfe von kleinen Stiften auf die Abdeckung drückt, um ein bestimmtes Muster zu erzeugen.
\subsection{Durchsichtiges LCD Display im Dachfenster}
An das Dachfenster wurde ein durchsichtiges LCD Display geklebt, das in schwarz weiß Bilder darstellen kann.
\subsection{LED Matrix im Dachhimmel}
Im Dachhimmel unter der Bestoffung befindet sich ein Matrix RGB LED Feld. Das Feld kann dynamische Farbeffekte für die Fahrzeugpassagiere erzeugen.
\section{Ansteuerung}
