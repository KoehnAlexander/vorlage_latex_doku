\chapter{Konzeptentwurf für eine elektrische und informationstechnische Anbindung der Komponenten} \label{cha:Konzeptentwurf}
Auf Basis der unter \ref{cha:Analyse} gewonnenen Erkenntnissen zum Strombedarf und Informationsbedarf der Komponenten werden in diesem Kapitel Entwürfe erarbeitet, wie die Komponenten in eine Fahrzeugarchitektur implementiert werden können.
Im ersten Schritt werden die durch die Analyse gewonnen Erkenntnisse genutzt, um Komponenten nach ihren Eigenschaften zu sortieren.
Abweichend von der Betrachtungsweise der verwendeten Technologie und des Einbauortes, ist es sinnvoll nach den Art der Informationsbedarfes zu unterteilen.
Die Anforderungen an die Komponenten sind gleich. Jede Komponente muss in Echtzeit auf mindestens 10 Kollektionen umschalten können.

Eine Komponente besteht in diesem Fall immer aus einer Anzeigefläche mit einer dazugehörigen Steuerung, die aus Bild oder Videodateien die einzelnen Bildpunkte setzt.

Die Kollektionen werden über eine Drahtlose Verbindung zwischen einem zentralen Steuergerät aus dem Internet geladen und müssen auf den Komponenten im Fahrzeug dargestellt werden. 
Das zentrale Abspeichern der Daten und anzeigen der Daten bei Auswahl ist eine mögliche Option.
Die Daten können auch über das Fahrzeugnetzwerk vom zentralen Steuergerät auf verteilte Steuergeräte, die direkt an den Komponenten sind, geladen werden, wenn die Daten zu dem Zeitpunkt nicht angezeigt werden sollen. Bei Aktivierung der Kollektion müssen dann nur Steuersignale den Steuergeräten die Information zur Aussendung der Kollektion geben.
Relevant für die möglichen Optionen, sind die Latenz eines Steuersignals und der Dauer eines kompletten Downloads der Daten.
Der Anschluss der Komponenten an eine bestimmte Architektur ist unter unterschiedlichen Gesichtspunkten zu bewerten.

Es gibt unterschiedliche Varianten die Geräte anzuschliessen.

Die erste Form von Komponenten sind diejenigen, die nicht dafür ausgelegt sind neue Inhalte in Echtzeit vom zentralen System herunterzuladen.
Die zweite Form sind Komponenten, die in Echtzeit neu Inhalte vom System herunterladen müssen und direkt anzeigen.

Echtzeit bedeutet in diesem Fall, dass das System neue Inhalte anzeigt, die der Benutzer über das Zentraldisplay oder Drittgerät ausgewählt hat.
Durch diese Unterscheidung kann die zur Verfügung gestellte Bandbreite deutlich unterschiedlich sein. 
Ein weiterer wichtiges Merkmal ist die Geschwindigkeit zum 


Architekturdiagramm einer Möglichen Variante:
