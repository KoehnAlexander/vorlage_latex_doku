\chapter{Konzeptentwurf für eine elektrische und informationstechnische Anbindung der Komponenten} \label{cha:Konzeptentwurf}
Auf Basis der unter \ref{cha:Analyse} gewonnenen Erkenntnissen zum Strombedarf und Informationsbedarf der Komponenten werden in diesem Kapitel Entwürfe erarbeitet wie die Komponenten in eine Fahrzeugarchitektur implementiert werden können.
Im ersten Schritt werden die durch die Analyse gewonnen Erkenntnisse genutzt, um mögliche Komponenten neu zu sortieren.
Abweichend von der Betrachtungsweise der verwendeten Technologie und des Einbauortes, ist es sinnvoll nach den Art der Informationsaufnahme einzuteilen. 
Die erste Form von Komponenten sind diejenigen, die nicht dafür ausgelegt sind neue Inhalte in Echtzeit vom zentralen System herunterzuladen.
Die zweite Form sind Komponenten, die in Echtzeit neu Inhalte vom System herunterladen müssen und direkt anzeigen.

Echtzeit bedeutet in diesem Fall, dass das System neue Inhalte anzeigt, die der Benutzer über das Zentraldisplay oder Drittgerät ausgewählt hat, und noch nicht Im System hinterlegt war.
Durch diese Unterscheidung kann die zur Verfügung gestellte Bandbreite deutlich unterschiedlich sein. 
Ein weiterer wichtiges Merkmal ist die Geschwindigkeit zum 