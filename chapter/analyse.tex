\chapter{Analyse der Komponenten des Prototypen}
\label{cha:Analyse}
Nachfolgend werden alle Komponenten unter den von \ref{cha:Kriterien} genannten Kriterien analysiert. Im Zuge der Analyse werden im nächsten Kapitel Entwürfe für die Implementierung der Komponenten dargestellt.\\
Die Analyse basiert auf den Komponenten des Prototypen, aber beschränkt sich nicht auf die einzeln verbaute Technologie, sondern geht auf den Gesamtkontext der Komponente ein. Grundlegend richtet sich die Analyse nach dem sinnlichen Darstellungszweck und dessen Bedarf an technischen Komponenten. Ein Beispiel dieser Herangehensweise ist der Bildschirm in der Eintiegsleiste. Die genaue Größe einer möglichen Serienimplementierung muss nicht gleich des Prototypen sein, aber sie muss eine Größe haben, die für die Darstellung der Inhalte geeignet ist.\\
Die Kriterien für den Einbau und die spätere Analyse basieren zum Einen auf einer selbstständigen Recherche über die Komponenten und den relevanten Bedingungen. Zum Anderen wurden Informationen von Experten in den einzelnen Fachgebieten der Karosserieinnenentwicklung, Individualisierung, After-Sales-Produktentwicklung herangezogen.
\section{Kriterien für den Einbau der Komponenten in Serienfahrzeuge}
\label{cha:Kriterien}
Die folgenden Kriterien richten sich zum Einen an die Komponenten und zum Anderen an die Fahrzeugentwicklung. Anhand der Kriterien kann eine Analyse des Ist-Zustandes angewendet werden.\\
Da für den Verkauf und Betrieb von Fahrzeugen die gültigen Gesetze und Normen eingehalten werden müssen, haben die rechtlichen Kriterien die oberste Priorität. Wenn alle rechtlichen Bedingungen erfüllt sind, stellen sich betriebswirtschaftliche Fragestellungen, ob die Komponenten in Ihrer Funktion einen Nutzen für den Kunden und für das Unternehmen haben. Dies können hier besonders auch optische Vorteile sein. Zuletzt müssen technische Kriterien von der Seite der Fahrzeugentwicklung und der Komponenten erfüllt sein, damit die Weiterentwicklung der Komponenten in Erwägung gezogen werden kann. \\
Durch den zeitlichen Rahmen dieser Arbeit und dem fachlichen Schwerpunkt in elektrotechnischen Systemen, liegt der Fokus der Analyse in der technischen Analyse des Strombedarfs und des Informationsbedarfs. Dennoch soll diese Arbeit die wichtigsten Aspekte und eine Einordnung der anderen Fachbereiche liefern.\\
Nachfolgend werden mit der gleichen Reihenfolge wie oben die Kriterien definiert.
\subsection{Rechtliche Kriterien}
Unter rechtlichen Kriterien finden sich alle Bedingungen für die Fahrzeugentwicklung, die auf Basis von Gesetzen und Normen für eine Zulassung erfüllt werden müssen. Diese Kriterien gewährleisten zugleich Rechtssicherheit für den Hersteller. \\
Aus Gründen des Überblicks wird in der folgenden Arbeit nur auf einzelne Themen von rechtlichen Kriterien eingegangen. Für Lichter am Fahrzeug gilt die europäische Richtlinie ECE R48. Im Innenraum muss die Vermeidung von Fahrerablenkung sichergestellt werden.\\
Im Rahmen einer weiteren Entwicklung von Komponenten für eine Fahrzeugserie sind die rechtlichen Bedingungen unbedingt frühzeitig von internen oder externen Juristen und Juristinnen zu prüfen.
\subsection{Wirtschaftliche Kriterien}
Unter wirtschaftliche Kriterien fallen sowohl betriebswirtschaftliche Betrachtungen als auch kundenspezifische Anforderungen. Diese Kriterien sind Grundvoraussetzungen für eine Erstellung eines Geschäftsplan und sollen somit einen Rahmen in diesem Bereich bieten. Die weiteren Schritte nach Erfüllung dieser Kriterien sind Absprachen mit Produktstrategen, Marketingexperten und Kalkulatoren. \\
Unter dem betriebswirtschaftlichen Kriterium steht in erster Linie die Gewinnerzielungsabsicht, das heißt die Kosten des Produkts sollen geringer als die möglichen Einnahmen sein. Kosten können neben den Entwicklungskosten und Produktionskosten für die Komponente, auch Verwaltungskosten über den Lebenszyklus des Produkts und Anpassungskosten anderer Bauteile im Fahrzeug sein. \\
Einnahmen sind neben dem Verkauf der Komponente als Sonderausstattung weitere sekundäre Möglichkeiten durch digitale Produkte spezifisch zu den Komponenten. Daneben können es auch nicht monetäre Einnahmen geben, die das gesamte Produkt und die Marke in ihrem Ansehen stärken.\\
Um einen Gewinn zu erwirtschaften, muss die Komponente mit ihren Fixkosten wie die Entwicklung und ihren variablen Kosten wie den Produktionskosten gering sein, aber dennoch einen großen Mehrwert für den Kunden liefern, damit die Nachfrage an der Komponente steigt. \\
Die Entwicklungskosten sind abhängig vom aktuellen Reifegrad vorhandener Technik und der Komplexität der Technik, während Anpassungskosten abhängig sind von dem entstehenden Bedarf an Architekturveränderungen. Die Produktionskosten sind abhängig von den Einkaufspreisen der Komponenten und der Montage. Diese Kriterien werden in dieser Arbeit nach der technischen Analyse gesammelt betrachtet, um einen Vergleich mit den einzelnen Komponenten zu liefern. \\
Auf der Seite der Einnahmen steht im Mittelpunkt das Kundeninteresse für den Kauf solcher Komponenten. Kriterien für den Kauf von Komponenten durch Kunden können durch Probandenstudien gewonnen werden oder aus Erfahrung von ähnlichen Projekten herangezogen werden.
\subsection{Technische Kriterien}
Unter technischen Kriterien fallen alle relevanten Gebiete der Produktentwicklung:
\begin{enumerate}
	\item Physikalische Dimensionierung
	\item Stabilität
	\item Gewicht
	\item Elektrischer Energiebedarf
	\item Informationsbedarf
	\item Optik
	\item Wartungsfähigkeit
\end{enumerate}
\paragraph{Physikalische Dimensionierung}
Zur Erfolgreichen Integration der Komponente in eine Fahrzeugentwicklung muss an der gewünschten Stelle genügend Bauraum zur Verfügung stehen.
\paragraph{Stabilität}
Die Komponente muss an der Einbaustelle befestigt werden können. Alle Stabilität- und Crashtests müssen positiv ausgefallen sein.
\paragraph{Gewicht}
Das Fahrzeug muss das zusätzliche Gewicht der Komponenten an dieser Stelle aufnehmen können.
\paragraph{Elektrischer Energiebedarf}
Die Komponente soll einen möglichst geringen Strombedarf haben, während das Fahrzeug diesen an der Stelle zur Verfügung stellen muss.
\paragraph{Informationsbedarf}
Zur Steuerung der Komponente muss an der Stelle eine Möglichkeit vorhanden sein mit dem Gesamtfahrzeug zu kommunizieren. Kriterien sind zum einen die benötigte Bandbreite der Komponenten und die Geschwindigkeit des Informationsaustausches.
\paragraph{Optische Kriterien}
Unter optischen Kriterien fallen alle allgemein gültige Designprinzipien in der Produktentstehung. Darunter versteht sich die Harmonie der Komponenten in der Umgebung, die Vermeidung von Bildschirmrändern, usw.
\paragraph{Wartungsfähigkeit}
Unter Wartungsfähigkeit fällt der leichte Ein- und Ausbau der Komponente bei Bedarf und der Austausch von Verschleißgegenständen.

\section{Analyse der Exterieur Komponenten}
Im folgenden werden die Exterieur Komponenten an Hand der oben beschriebenen einzelnen Kriterien geprüft. Die wirtschaftlichen Kriterien werden in einem separaten Abschnitt gesammelt betrachtet. 
\subsection{E-Papier in der Frontschürze}
Rechtlich ist zu Prüfen, ob das E-Papier als Leuchte gilt und danach behandelt werden muss. Daneben muss rechtlich abgesichert werden, ob das Ändern der Inhalte nur im Abgestellten Fahrmodus erfolgen darf, oder auch bei Fahrt.\\
Wirtschaftlich betrachtet ist dies eine vollständige Neuentwickl
Um ein scharfes Bild für Betrachter aus ein bis zwei Metern Entfernung zu liefern, benötigt das E-Papier bei einer Größe mit einer Breite von 80 cm und einer Höhe von 60cm eine Auflösung von mindestens Full HD.
Das E-Papier sollte von der Distanz von zwei Metern ein scharfes Bild liefern. Mit der ungefähren Größe mit einer Breite von 80 cm und einer Höhe von 60cm ist das 
Durch die Position ist die gezielte Brechung bei Crashs relevant für die Sicherheit von Passanten.
Die Stromversorgung muss durch eine Verkabelung im Exterieur sichergestellt werden. Dazu benötigt das E-Papier eine Leistung von $ X\,\mathrm{W} $ .
Daneben ist zu Prüfen, ob die Radarsensorik von dem E-Papier gestört wird.\\
Wirtschaftlich betrachtet ist dies eine vollständige Neuentwicklung an der der Kotflügel und die Frontlichter mit angepasst werden müssen. Aus Kundensicht kann das E-Papier die Front des Fahrzeug in seiner Optik deutlich verändern. 
\subsection{LED-Streifen in der Frontschürze}
Unter den rechtlichen Kriterien gilt bei Lichtern die ECE R48. Farbliche Lichter außerhalb der Zulässigen Verbauten sind dabei nicht gestattet. Eine 
Daneben ist die pixeldichte.\\

Die LED-Streifen benötigen mit einer Anzahl von 500 LED eine Leistung von $ X\,\mathrm{W} $. Die LED-Streifen benötigen einen Anschluss an das Fahrzeugnetzwerk mit einer Bandbreite von $ X\,\frac{\mathrm{ KBit}}{\mathrm{s}} $
Daneben muss eine Ansteuerungslogik dort verbaut werden.
\subsection{E-Papier Embleme über den vorderen Radkästen}
Die E-Papier Embleme benötigen mit einer Auflösung von HD 
Herausforderung: Bauteil Bestromung im Exterierunr
Folie besteht DBL-Testing nicht 
\subsection{LED-Streifen in den Radkästen}
Die Verschmutzung des Lichtleiters ist ein Kriterium.
Daneben muss aus Kundensicht geprüft werden, ob die Betonung des Schmutzfänger Bereichs gewünscht ist, oder nur in besten Situationen.
Daneben bestehen wie bei den anderen Licht Leisten im Exterieur Marktregularien wegen der farbigen Beleuchtung. Die Vernetzung an dieser Stelle ist eine Herausforderung.
\subsection{Videoprojektoren in den Außenspiegeln}
Die Verfügbarkeit des Bauraums in dem Außenspiegel ist zu prüfen. Daneben ist die Zulässigkeit von solchen Projektionen und deren Lichtaustrittsfläche rechtlich abzuprüfen.
\subsection{Bildschirme in den hinteren Seitenfenstern}
Rechtlich ist die Zulässigkeit wegen möglicher Ablenkungen von Straßenverkehrsteilnehmern zu prüfen.
Der Bauraum an dieser Stelle und das Verkleinern der Fenster Fläche ist abzuprüfen.
Die Crashsicherheit an dieser Stelle ist abzuprüfen.

Vernetzung
\subsection{LED-Streifen in der Heckleuchte}
Herausforderungen:
Wie bei Frontlichtband im Scheinwerfer
Bereitstellung Vernetzungsschnittstelle und Ansteuerungslogik

\subsection{E-Papier in der Heckleuchte}

\section{Analyse der Interieur Komponenten}
\subsection{LED-Streifen im Interieur}
Herausforderung:
Ansteuerung
\subsection{Matrix LED Türtafeln}
Rechtlich gesehen ist zu prüfen, ab welchem Helligkeitsgrad und Farbinszenierungsdynamik der Fahrer oder die Fahrerin durch die Matrix LED zu stark abgelenkt werden.\\
Die Matrix LED benötigt bei der aktuellen 
\subsection{Bildschirme in der Einstiegsleiste}

Die technische Umsetzbarkeit fiel in Voruntersuchungen (wie Kugelfalltest, Bauraum, Shaker) positiv aus.
Kriterien an dieser Stelle sind Feuchtigkeiten, Temperaturen, und die Kratzfestigkeit.
Da an dieser Stelle eine Echtzeitfähigkeit gegeben sein muss
Feuchtigkeit, Temperatur, Kratzfest
Datensicherheit, Schnittstellen
Fahrzeuganbindung, Ansteuerung

\subsection{Videoprojektoren im Fußraum}
Herausforderung:
Bauraum, Wackeln im Fahrbetrieb, Temperaturentwicklung
\subsection{Morphende Oberfläche in der Mittelkonsole}
Herausforderung:
Materialfestigkeit, Spannungen,

\subsection{Durchsichtiger Bildschirm im Dachfenster}
Herausforderung:
Regularien zu licht Transmission bei Nacht nach außen
Verbindung/ Kontaktierung
Temperatur
\subsection{LED Matrix im Dachhimmel}
Herausforderungen:
Absicherung/ Crash Test, ideale Verbauposition
Hitzeentwicklung auf Dach
Ansteuerung, Montage/ Konstruktion
Lichtleiterkonzepte
\subsection{Duftflakons im Innenraum}
\subsection{Bildschirmoberflächen im Cockpit}
\subsection{Soundplayer im Innenraum}
\section{Gesamtwirtschaftliche Analyse}
In diesem Abschnitt werden die Komponenten als Gesamtes unter den wirtschaftlichen Kriterien betrachtet, weil die Komponenten nicht als einzelne Sonderausstattung, sondern als Ganzes oder Gebündelt vertrieben werden soll. Daher ist eine Gesamtbetrachtung der wirtschaftlichen Kriterien sinnvoll. \\
In Tabelle \ref{tab:Entwicklung} werden alle Komponenten mit den nötigen Anpassungen in der Serienentwicklung in den bestimmten Bereichen auf Basis der vorherigen technischen Analyse. 
\begin{table}[hbt]	
	\centering
	\renewcommand{\arraystretch}{1.5}	% Skaliert die Zeilenhöhe der Tabelle
	\captionabove[Liste der Komponenten mit den nötigen Eingriffen in die Fahrzeugentwicklung]{Liste der Komponenten mit den nötigen Eingriffen in die Fahrzeugentwicklung}
	\label{tab:Entwicklung}
	\begin{tabular}{c|ccc}
		\textbf{Komponente} & \textbf{Karosserie} & \textbf{E/E Architektur} & \textbf{Software} \\ 
		\hline 
		\hline 
		E-Papier in der Frontschürze & X & X & X \\
		LED-Streifen in der Frontschürze & X & X & X \\
		E-Papier Embleme über den vorderen Radkästen & X & X & X \\
		LED-Streifen in den Radkästen & X & X & X \\
		Videoprojektoren in den Außenspiegeln & X & X & X \\
		Bildschirme in den hinteren Seitenfenstern & X & X & X \\
		LED-Streifen in der Heckleuchte &  & X & X \\
		E-Papier in der Heckleuchte & X & X & X \\
		LED-Streifen im Interieur &  & X & X \\
		Matrix LED Türtafeln &  & X & X \\
		Bildschirme in der Einstiegsleiste &  & X & X \\
		Videoprojektoren im Fußraum &  & X & X \\
		Morphende Oberfläche in der Mittelkonsole &  & X & X \\
		Durchsichtiger Bildschirm im Dachfenster &  & X & X \\
		LED Matrix im Dachhimmel & X & X & X \\
		Bildschirmoberflächen im Cockpit &  &  & X \\
		Soundplayer im Innenraum &  & X & X \\
		Duftflakons im Innenraum &  & X & X \\
	\end{tabular} 
\end{table}
		%Der parbox-Befehl ist erforderlich, damit ein Zeilenumbruch erzeugt werden kann. c-Spalten (zentriert) erlauben nicht automatisch einen Zeilenumpruch. Linksbündig gesetzte p-Spalten erlauben automatisch den Zeilenumbruch.%

