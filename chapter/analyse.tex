\chapter{Analyse der Komponenten des Prototypen}
\label{cha:Analyse}
Nachfolgend werden alle Komponenten unter den von \ref{cha:Kriterien} genannten Kriterien analysiert. Im Zuge der Analyse werden im nächsten Kapitel Entwürfe für die Weiterentwicklung der Komponenten dargestellt.\\
Die Analyse basiert auf den Komponenten des Prototypen, aber beschränkt sich nicht auf die einzeln verbaute Technologie, sondern geht auf den Gesamtkontext der Komponente ein. Grundlegend richtet sich die Analyse nach dem sinnlichen Darstellungszweck und dessen Bedarf an technischen Komponenten. \\
Ein Beispiel dieser Herangehensweise ist der Bildschirm in der Einstiegsleiste. Die genaue Größe einer möglichen Serienimplementierung muss nicht gleich des Prototypen sein, aber sie muss eine Größe haben, die für die Darstellung der Inhalte geeignet ist.\\
Die Kriterien für den Einbau und die spätere Analyse basieren zum Einen auf einer selbstständigen Recherche über die Komponenten und den relevanten Bedingungen. Zum Anderen wurden Informationen von Experten in den einzelnen Fachgebieten der Karosserieinnenentwicklung, Individualisierung, After-Sales-Produktentwicklung herangezogen, um die Kriterien im Bereich rechtlich, wirtschaftlich und technische anzuwenden. \\
Für einen Überblick über die Erkenntnisse fasst das Unterkapitel XX die Analyse zusammen
\section{Kriterien für den Einbau der Komponenten in Serienfahrzeuge}
\label{cha:Kriterien}
Die folgenden Kriterien richten sich zum Einen an die Komponenten und zum Anderen an die Fahrzeugentwicklung. Anhand der Kriterien kann eine Analyse des Ist-Zustandes angewendet werden.\\
Da für den Verkauf und Betrieb von Fahrzeugen die gültigen Gesetze und Normen eingehalten werden müssen, haben die rechtlichen Kriterien die oberste Priorität. Wenn alle rechtlichen Bedingungen erfüllt sind, stellen sich betriebswirtschaftliche Fragestellungen, ob die Komponenten in Ihrer Funktion einen Nutzen für den Kunden und für das Unternehmen haben. Dies können hier besonders auch optische Vorteile sein. Zuletzt müssen technische Kriterien von der Seite der Fahrzeugentwicklung und der Komponenten erfüllt sein, damit die Weiterentwicklung der Komponenten in Erwägung gezogen werden kann. \\
Durch den zeitlichen Rahmen dieser Arbeit und dem fachlichen Schwerpunkt in elektrotechnischen Systemen, liegt der Fokus der Analyse in der technischen Analyse des Strombedarfs und des Informationsbedarfs. Dennoch soll diese Arbeit die wichtigsten Aspekte und eine Einordnung der anderen Fachbereiche liefern.\\
Nachfolgend werden mit der gleichen Reihenfolge wie oben die Kriterien definiert.
\subsection{Rechtliche Kriterien}
Unter rechtlichen Kriterien finden sich alle Bedingungen für die Fahrzeugentwicklung, die auf Basis von Gesetzen und Normen für eine Zulassung erfüllt werden müssen. Diese Kriterien gewährleisten zugleich Rechtssicherheit für den Hersteller. \\
Aus Gründen des Überblicks wird in der folgenden Arbeit nur auf einzelne Themen von rechtlichen Kriterien eingegangen. 
Beispiele für rechtliche Kriterien sind für Lichter am Fahrzeug die europäische Richtlinie ECE R48. Im Innenraum muss die Vermeidung von Fahrerablenkung sichergestellt werden.\\
Im Rahmen einer weiteren Entwicklung von Komponenten für eine Fahrzeugserie sind die rechtlichen Bedingungen unbedingt frühzeitig von internen oder externen Juristen und Juristinnen zu prüfen.
\subsection{Wirtschaftliche Kriterien}
Unter wirtschaftliche Kriterien fallen sowohl betriebswirtschaftliche Betrachtungen als auch kundenspezifische Anforderungen. Diese Kriterien sind Grundvoraussetzungen für eine Erstellung eines positiven Geschäftsplan und sollen somit einen Rahmen in diesem Bereich bieten. Die weiteren Schritte nach Erfüllung dieser Kriterien sind Absprachen mit Produktstrategen, Marketingexperten und Kalkulatoren. \\
Unter dem betriebswirtschaftlichen Kriterium steht in erster Linie die Gewinnerzielungsabsicht, das heißt die Kosten des Produkts sollen geringer als die möglichen Einnahmen sein. Kosten können neben den Entwicklungskosten und Produktionskosten für die Komponente, auch Verwaltungskosten über den Lebenszyklus des Produkts und Anpassungskosten anderer Bauteile im Fahrzeug sein. \\
Einnahmen sind neben dem Verkauf der Komponente als Sonderausstattung weitere sekundäre Möglichkeiten durch digitale Produkte spezifisch zu den Komponenten. Daneben können es auch nicht monetäre Einnahmen geben, die das gesamte Produkt und die Marke in ihrem Ansehen stärken.\\
Um einen Gewinn zu erwirtschaften, muss die Komponente mit ihren Fixkosten wie die Entwicklung und ihren variablen Kosten wie den Produktionskosten gering sein, aber dennoch einen großen Mehrwert für den Kunden liefern, damit die Nachfrage an der Komponente steigt. \\
Die Entwicklungskosten sind abhängig vom aktuellen Reifegrad vorhandener Technik und der Komplexität der Technik, während Anpassungskosten abhängig sind von dem entstehenden Bedarf an Architekturveränderungen. Die Produktionskosten sind abhängig von den Einkaufspreisen der Komponenten und der Montage. \\
Auf der Seite der Einnahmen steht im Mittelpunkt das Kundeninteresse für den Kauf solcher Komponenten. Kriterien für den Kauf von Komponenten durch Kunden können durch Probandenstudien gewonnen werden oder aus Erfahrung von ähnlichen Projekten herangezogen werden. \\
Diese wirtschaftlichen Kriterien werden in dieser Arbeit nach der technischen Analyse gesammelt betrachtet. Das Gesamtkonzept für das Fahrzeug fordert eine ganzheitliche Inszenierung des Fahrzeugs, wofür mehrere Komponenten benötigt werden, um diesen Effekt zu erzielen. Aus diesem Hintergrund ist es sinnvoll die Entwicklung der Komponenten ganzheitlich wirtschaftlich zu bewerten. \\
\subsection{Technische Kriterien}
Unter technischen Kriterien fallen alle relevanten Gebiete der Produktentwicklung:
\begin{enumerate}
	\item Physikalische Dimensionierung
	\item Stabilität
	\item Gewicht
	\item Elektrischer Energiebedarf
	\item Informationsbedarf
	\item Optik
	\item Wartungsfähigkeit
\end{enumerate}
Je nach Beschaffenheit der Komponenten können diese Gebiete unterschiedlich ins Gewicht fallen.
\paragraph{Physikalische Dimensionierung}
Zur Erfolgreichen Integration der Komponente in eine Fahrzeugentwicklung muss an der gewünschten Stelle genügend Bauraum zur Verfügung stehen.
\paragraph{Stabilität}
Die Komponente muss an der Einbaustelle befestigt werden können. Alle Stabilität- und Crashtests müssen positiv ausgefallen sein.
\paragraph{Gewicht}
Das Fahrzeug muss das zusätzliche Gewicht der Komponenten an dieser Stelle aufnehmen können.
\paragraph{Elektrischer Energiebedarf}
Die Komponente soll einen möglichst geringen Strombedarf haben, während das Fahrzeug diesen an der Stelle zur Verfügung stellen muss.
\paragraph{Informationsbedarf}
Zur Steuerung der Komponente muss an der Stelle eine Möglichkeit vorhanden sein mit dem Gesamtfahrzeug zu kommunizieren. Kriterien sind zum einen die benötigte Bandbreite der Komponenten und die Geschwindigkeit des Informationsaustausches. \\
In der folgenden Arbeit wird bei Videos mit einer Frame-Rate von 24 Bilder pro Sekunde ausgegangen und einer Farbtiefe pro Farbe von einem Byte. Bei Bildern wird ein Kompressionsfaktor von zwölf genutzt. Wie bei \ref{cha:Prototyp} erwähnt sollen zehn Kollektionen mit fünf Inhalten im System gespeichert sein. Ein Inhalt kann ein Bild oder ein Video mit einer Länge von zehn Sekunden sein.
Die Berechnung des Speicherbedarfes dient als erste Abschätzung für die Speicherdimensionierung und kann vom tatsächlichen Bedarf abweichen, da die Rechnung Prüf- und Steuerdaten nicht beachtet.
\paragraph{Optische Kriterien}
Unter optischen Kriterien fallen alle allgemein gültige Designprinzipien in der Produktentstehung. Darunter versteht sich die Harmonie der Komponenten in der Umgebung, die Vermeidung von Bildschirmrändern, usw.
\paragraph{Wartungsfähigkeit}
Unter Wartungsfähigkeit fällt der leichte Ein- und Ausbau der Komponente bei Bedarf und der Austausch von Verschleißgegenständen.
\section{Analyse der Exterieur Komponenten}
Im folgenden werden die Exterieur Komponenten an Hand der oben beschriebenen einzelnen Kriterien geprüft. Die wirtschaftlichen Kriterien werden in einem separaten Abschnitt gesammelt betrachtet. 
\subsection{E-Papier in der Frontschürze}
Rechtlich ist zu Prüfen, ob das E-Papier als Leuchte gilt und danach behandelt werden muss. Daneben sind die rechtlichen Grundlagen für das Ändern der Bildschirminhalten in unterschiedlichen Fahrmodi, wie zum Beispiel Parken oder Fahren, zu analysieren. \\
Wirtschaftlich betrachtet ist diese Komponente ein zentraler Bestandteil des Fahrzeugkonzeptes, da durch die Größe und Lage der Kunde die Inhalte stark wahrnimmt. \\
Mit einer physikalischen Dimensionierung über die gesamte Breite zwischen den Frontlichtscheinwerfern und somit eine mögliche Breite bei unterschiedlichen Fahrzeugmodellen von 80 cm bis 1 m und einer Höhe von 50 cm bis 60 cm ist die Stabilität des E-Papiers zu prüfen, da die Lage im Kotflügel besonders transponiert ist. \\ 
Das zusätzliche Gewicht des Bildschirms ist an der vorderen Kotflügelbefestigung aufzunehmen.\\
Der Strombedarf des E-Papier ist, wie in \ref{cha:Grundlagen} beschrieben, nur gering. Ein Mikrocontroller zur Steuerung benötigt zum Beispiel eine Spannung von 12V Gleichstrom. \\
Mit einer Auflösung von 2560 Pixel pro Zeile zu 1440 Zeilen und einer Farbtiefe von 16 Stufen benötigt das Display mit der Formel \ref{eq:Bilddatenmenge} pro Bild einen Speicher S von $ 307,2\,\mathrm{kByte} $. 
\begin{align}
	S &= \frac{2560\,\frac{\mathrm{Pixel}}{\mathrm{Zeile}}\times 1440\,\mathrm{Zeilen} \times 8\,\frac{\mathrm{Bit}}{\mathrm{Pixel}}}{12} \\
	&= 2.457.600\,\mathrm{Bit} = 2,2567\,\mathrm{MBit} = 307,2\,\mathrm{KByte}
\end{align}
Bei zehn Kollektionen mit jeweils fünf Bildern benötigt man einen Speicher von $ 15,36\,\mathrm{MByte} $. \\
Durch die Position ist die gezielte Brechung bei Crashs relevant für die Sicherheit von Passanten.
Daneben ist zu Prüfen, ob die Radarsensorik von dem E-Papier gestört wird.\\
\subsection{LED-Streifen in der Frontschürze}
Unter den rechtlichen Kriterien gilt bei Lichtern die europäische Richtlinie ECE R48. Farbliche Lichter außerhalb der zulässigen Verbauten sind dabei nicht gestattet. Daher gilt es zu prüfen, welche weitere Möglichkeiten dieser Darstellungsart es gibt. Optionen sind beispielsweise schwarz-weiße Lichter oder farbliche Bildschirme ohne aktive Beleuchtung. \\
Für eine Farbtiefe von 256 Stufen pro Farbe benötigt jede einzelne LED 24 Bit an Speicher, da es drei mal acht Bit benötigt. Bei 332 LEDs bedeutet das einen Speicher pro Bild von circa einem Kilobit. 
\begin{align}
	S &= \frac{332\,\mathrm{Pixel} \times 24\,\frac{\mathrm{Bit}}{\mathrm{Pixel}}}{12} \cdot \\
	&= 1.000\,\mathrm{Bit} = 1\,\mathrm{kBit} = 125\,\mathrm{Byte}
\end{align}
Für ein 10 Sekunden Video wird ein Speicher S von 
\begin{align}
	S &= 1\,\mathrm{kBit} \cdot 24\,\mathrm{fps} \cdot 10\,\mathrm{s}\\
	&= 240\,\mathrm{kBit} = 30\,\mathrm{kByte}
\end{align}
benötigt. Bei insgesamt 50 Inhalten bedeutet das $ 1,5\,\mathrm{MByte} $ an Speicherbedarf. \\
Eine farbige LED, die pro Pixel vier Dioden (Rot, Grün, Blau und Weiß) enthält, hat einen Leistungsaufnahme von $ 320\,\mathrm{mW} $.  
Die LED-Streifen benötigen mit einer Anzahl von 500 LED eine Leistung von $ 160\,\mathrm{W} $. Im 12 V Bordnetz beträgt die Stromstärke $ 13,3\,\mathrm{A} $. 
Die LED-Streifen benötigen einen Anschluss an das Fahrzeugnetzwerk mit einer Bandbreite von $ X\,\frac{\mathrm{KBit}}{\mathrm{s}} $ \\
Daneben muss eine Ansteuerungslogik dort verbaut werden.
\subsection{E-Papier Embleme über den vorderen Radkästen}
Die E-Papiere an den Fahrzeugseiten müssen, wie oben erwähnt, geprüft werden, ob sie als Beleuchtung gelten.\\
Aus Sicht der Kunden ist die Lage neben den Fronttüren ideal, um Dinge anzuzeigen.
Da auf diesen E-Papieren vorwiegend Text angezeigt wird, ist ein breites und in der Höhe schmales Display von Vorteil, da dort eine Zeile leserlich angezeigt werden kann. \\
Um bei Beschädigung oder bei Defekt das E-Papier auszutauschen, ist die Möglichkeit über den Zugriff vom Motorraum aus zu prüfen. \\
Das Gewicht des E-Papiers ist relativ gering und zu vernachlässigen.
Da das E-Papier nur bei Änderung des Inhaltes Strom verbraucht, muss nur eine geringe Leistungsaufnahme von $ x\,\mathrm{mW} $ sichergestellt werden. \\
Die E-Papier Embleme benötigen mit einer Auflösung von 1600 Pixel pro Zeile zu 1200 Zeilen einen Speicher pro Bild von $ 160\,\mathrm{kByte} $.
\begin{align}
	S &= \frac{1600\,\frac{\mathrm{Pixel}}{\mathrm{Zeile}} \times 1200\,\mathrm{Zeilen} \times 8\,\frac{\mathrm{Bit}}{\mathrm{Pixel}}}{12} \cdot \\
	&= 1.280.000\,\mathrm{Bit} = 1,28\,\mathrm{MBit} = 160\,\mathrm{kByte}
\end{align}
Insgesamt benötigt der Speicher eine Größe von $ 8\,\mathrm{MByte} $ für 50 Bilder.
\subsection{LED-Streifen in den Radkästen}
Wie bei den anderen Leuchteinrichtungen im Exterieur sind hier die Marktregularien zu prüfen. Bei Möglichkeit kann auch ein eingeschränkter Modus von schwarz-weißem Licht oder nur die Anzeige im abgestellten Zustand des Fahrzeuges gewählt werden. \\
Die Verschmutzung des Lichtleiters ist durch eine geeignete Konstruktion zu verhindern und daneben auch der Beschädigung durch geschleuderte Steine oder ähnliches. \\
Daneben muss aus Kundensicht geprüft werden, ob die Betonung des Schmutzfänger Bereichs gewünscht ist, oder nur in bestimmten Situationen.
Bei einer Anzahl von 200 LED pro Radkasten benötigt ein LED Streifen eine Leistung von $ 64\,\mathrm{W} $.
Ein dynamisches Lichtspiel für 10 Sekunden benötigt pro Radkasten bei einer Bildwiederholungsrate von $ 24\,\mathrm{fps} $ einen Speicher von $ 12\,\mathrm{kByte}$. \\
\begin{align}
	S &= \frac{200\,\mathrm{Pixel} \times 24\,\frac{\mathrm{Bit}}{\mathrm{Pixel}}}{12} \cdot \\
	&= 400\,\mathrm{Bit} = 50\,\mathrm{Byte}
\end{align}
\begin{align}
	S &= 400\,\mathrm{Bit} \cdot 24\,\mathrm{fps} \cdot 10\,\mathrm{s}\\
	&= 96\,\mathrm{kBit} = 12\,\mathrm{kByte}
\end{align}
Für 50 Videos beträgt die Speichergröße $ 600\,\mathrm{kByte} $.
Die Vernetzung an dieser Stelle ist eine Herausforderung.
\subsection{Videoprojektoren in den Außenspiegeln}
Die Verfügbarkeit des Bauraums in dem Außenspiegel ist zu prüfen, indem der Bauraum von aktuellen Standbildprojektoren mit den benötigten Videoprojektoren verglichen wird. 
Daneben ist die Zulässigkeit von solchen Projektionen und deren Lichtaustrittsfläche rechtlich abzuprüfen. \\
Die angestrahlte Fläche auf dem Boden reicht im besten Fall über die gesamten Seitentüren und bis zu $ 2\,\mathrm{m} $ weg vom Fahrzeug.
Der Betrachter soll bei unterschiedlichen Lichtverhältnissen noch ein möglichst kontrastreiches und helles Bild erkennen. \\
Der Strombedarf pro Videoprojektor liegt im Bereich zwischen $ XX\,\mathrm{W} $ und $ XX\,\mathrm{W} $.
Ein Standbild benötigt bei einer Pixelanzahl pro Zeile von $ 1280 $ und $ 800 $ Zeilen ein Speichergröße von ca. $ 256\,\mathrm{kByte}$. 
\begin{align}
	S &= \frac{1280\,\frac{\mathrm{Pixel}}{\mathrm{Zeile}} \times 800\,\mathrm{Zeilen} \times 24\,\frac{\mathrm{Bit}}{\mathrm{Pixel}}}{12} \\
	&= 2.048.000\,\mathrm{Bit} = 2,048\,\mathrm{MBit} = 256\,\mathrm{KByte}
\end{align}
Für eine Animation von zehn Sekunden Länge benötigt eine Videoprojektor einen Speicher von $ 20,48\,\mathrm{GByte}$.
\begin{align}
	S &= 2,048\,\mathrm{kBit} \cdot 24\,\mathrm{fps} \cdot 10\,\mathrm{s}\\
	&= 491,52\,\mathrm{MBit} = 61,44\,\mathrm{MByte}
\end{align}
\subsection{Bildschirme in den hinteren Seitenfenstern}
Rechtlich ist die Zulässigkeit wegen möglicher Ablenkungen von Straßenverkehrsteilnehmern zu prüfen.
Das Verkleinern der Fensterfläche ist abzuprüfen, ob dies mit Kundeninteressen vereinbar ist.
Die Crashsicherheit an dieser Stelle und die Stabilität ist wichtig, da sich diese Komponente. in unmittelbarer Nähe zu den Fahrzeuginsassen befindet.
Jedes Display hat eine ungefähre Leistung von $ XX\,\mathrm{W} $. 
Ein Standbild benötigt bei einer Pixelanzahl pro Zeile von $ 1280 $ und von $ 800 $ Zeilen ein Speichergröße von ca. $ 256\,\mathrm{kByte}$. 
\begin{align}
	S &= \frac{1280\,\frac{\mathrm{Pixel}}{\mathrm{Zeile}}\times 800\,\mathrm{Zeilen} \times 24\,\frac{\mathrm{Bit}}{\mathrm{Pixel}}}{12} \\
	&= 2.048.000\,\mathrm{Bit} = 2,048\,\mathrm{MBit} = 256\,\mathrm{KByte}
\end{align}
Für eine Animation von zehn Sekunden Länge benötigt das Display einen Speicher von $ 61,44\,\mathrm{MByte}$.
\begin{align}
	S &= 2,048\,\mathrm{kBit} \cdot 24\,\mathrm{fps} \cdot 10\,\mathrm{s}\\
	&= 491,52\,\mathrm{MBit} = 61,44\,\mathrm{MByte}
\end{align}
Vernetzung
\subsection{LED-Streifen in der Heckleuchte}
Wie bei den LED-Streifen in der Frontschürze sind die rechtlichen und wirtschaftlichen Fragestellungen gleich. \\
Grundsätzlich gilt es zu prüfen, ob die vorhandenen Steuergeräte für die Lichter Kapazitäten für dynamische Lichtstreifen frei haben.
Die weiteren technischen Kriterien verhalten sich wie oben erläutert. \\
Für ein Bild benötigt man einen Speicher von $ 98\,\mathrm{Byte} $.
\begin{align}
	S &= \frac{391\,\mathrm{Pixel} \times 24\,\frac{\mathrm{Bit}}{\mathrm{Pixel}}}{12} \cdot \\
	&= 782\,\mathrm{Bit} = 98\,\mathrm{Byte}
\end{align}
Für eine zehn Sekunden Inszenierung sind $ 23,46\,\mathrm{kByte} $ erforderlich.
\begin{align}
	S &= 782\,\mathrm{Bit} \cdot 24\,\mathrm{fps} \cdot 10\,\mathrm{s}\\
	&= 187,68\,\mathrm{kBit} = 23,46\,\mathrm{kByte}
\end{align}
\subsection{E-Papier in der Heckleuchte}
Ähnliche Anforderungen wie die E-Papier Embleme über den vorderen Radkästen haben diese E-Papiere in der Heckleuchte
Diese E-Papiere in der Heckleuchte haben ähnliche Eigenschaften, wie die E-Papier Embleme über den vorderen Seitenkästen, in Bezug auf die rechtlichen Kriterien. Durch die Position in der Nähe der Heckleuchten kann von dort die Stromversorgung und Busanbindung bezogen werden. Der Strombedarf und Datenspeicherbedarf von $ 160\,\mathrm{kByte} $ pro Bild sind gleich mit dem anderen E-Papier. 
\begin{align}
	S &= \frac{1600\,\mathrm{Pixel pro Zeile} \times 1200\,\mathrm{Zeilen} \times 8\,\frac{\mathrm{Bit}}{\mathrm{Pixel}}}{12} \cdot \\
	&= 1.280.000\,\mathrm{Bit} = 1,28\,\mathrm{MBit} = 160\,\mathrm{kByte}
\end{align}
\section{Analyse der Interieur Komponenten}
Im folgenden werden die einzelnen Interieur Komponenten unter den Kriterien analysiert.
\subsection{LED-Streifen im Interieur}
Rechtlich ist bei den Animationen im LED-Streifen die Vermeidung von Ablenkung vom Fahrer oder Fahrerin zu gewährleisten.
Durch die Länge des Streifens muss für die Crashsicherheit eventuelle Sollbruchstellen sichergestellt werden. \\
Bei einer Anzahl von 719 LED pro Radkasten benötigt ein LED Streifen eine Leistung von $ XX\,\mathrm{W} $.
Für eine Inszenierung von zehn Sekunden Länge benötigt der LED-Streifen Speicherplatz in der Höhe von $ \,\mathrm{kByte}$.
\begin{align}
	S &= \frac{719\,\mathrm{Pixel} \times 24\,\frac{\mathrm{Bit}}{\mathrm{Pixel}}}{12} \cdot \\
	&= 400\,\mathrm{Bit} = 50\,\mathrm{Byte}
\end{align}
\begin{align}
	S &= 400\,\mathrm{Bit} \cdot 24\,\mathrm{fps} \cdot 10\,\mathrm{s}\\
	&= 96\,\mathrm{kBit} = 12\,\mathrm{kByte}
\end{align}
Für 50 Videos beträgt die Speichergröße $ 600\,\mathrm{kByte} $.
\subsection{LED Türtafeln}
Rechtlich gesehen ist zu prüfen, ab welchem Helligkeitsgrad und Farbinszenierung der Fahrer oder die Fahrerin durch die LED zu stark abgelenkt werden. \\
Im Moment wird nur eine LED pro Türtafel genutzt, wodurch Möglichkeiten der Inszenierung im Vergleich zu einer LED-Matrix nicht genutzt werden. 
Die LEDs benötigen eine Leistung von $ XX\,\mathrm{W} $ pro Türe.
Die Ansteuerung der Türtafeln kann durch die Nähe zu den LED-Streifen im Interieur zusammen gesteuert werden.
\subsection{Bildschirme in der Einstiegsleiste}
Unter den rechtlichen Aspekten gibt es in einer ersten Untersuchung keine Beschränkungen, da der Bildschirm nur bei geöffneter Türe und dementsprechend im Stand betrachtet werden kann.
Die technische Umsetzbarkeit ist durch Bestätigung in ersten Voruntersuchungen (wie Kugelfalltest, Bauraum, Shaker) positiv.
Technische Herausforderungen an dieser Stelle sind Feuchtigkeitsschutz, Temperaturempfindlichkeit, und die Kratzfestigkeit. \\
Jedes Display hat eine ungefähre Leistung von $ XX\,\mathrm{W} $. 
Ein Standbild benötigt bei einer Pixelanzahl pro Zeile von $ 1280 $ und von $ 1024 $ Zeilen ein Speichergröße von ca. $ 256\,\mathrm{kByte}$. 
\begin{align}
	S &= \frac{1280\,\frac{\mathrm{Pixel}}{\mathrm{Zeile}}\times 1024\,\mathrm{Zeilen} \times 24\,\frac{\mathrm{Bit}}{\mathrm{Pixel}}}{12} \\
	&= 2.048.000\,\mathrm{Bit} = 2,048\,\mathrm{MBit} = 256\,\mathrm{KByte}
\end{align}
Für eine Animation von zehn Sekunden Länge benötigt das Display einen Speicher von $ 61,44\,\mathrm{MByte}$.
Insgesamt bei 50 Videos beträgt der benötigte Speicher $ 61,44\,\mathrm{MByte}$.
\subsection{Videoprojektoren im Fußraum}
Unter den rechtlichen Fragestellungen ergibt sich die Vermeidung von Ablenkung für den Fahrer oder die Fahrerin.
Aus Kundenperspektive ist die angestrahlte Oberfläche eventuell nicht geeignet, da diese stärker verschmutzt sein kann. \\
Unter den technischen Kriterien ist die Verfügbarkeit des Bauraums mit der benötigten Kühlung zu prüfen. Eine Herausforderung ist die Befestigung des Videoprojektors, da dieser sich bei Fahrt relativ zum Auto nicht bewegen darf.
Jeder Videoprojektor hat eine ungefähre Leistung von $ XX\,\mathrm{W} $. 
Ein Standbild benötigt bei einer Pixelanzahl pro Zeile von $ 1280 $ und von $ 800 $ Zeilen ein Speichergröße von ca. $ 256\,\mathrm{kByte}$. 
\begin{align}
	S &= \frac{1280\,\frac{\mathrm{Pixel}}{\mathrm{Zeile}}\times 800\,\mathrm{Zeilen} \times 24\,\frac{\mathrm{Bit}}{\mathrm{Pixel}}}{12} \\
	&= 2.048.000\,\mathrm{Bit} = 2,048\,\mathrm{MBit} = 256\,\mathrm{KByte}
\end{align}
Für eine Animation von zehn Sekunden Länge benötigt das Display einen Speicher von $ 61,44\,\mathrm{MByte}$.
Insgesamt bei 50 Videos beträgt der benötigte Speicher $ 61,44\,\mathrm{MByte}$.
\subsection{Morphende Oberfläche in der Mittelkonsole}
Die tatsächlich verwendete technische Realisierung der morphenden Oberfläche ist eine Übergangslösung für eine neue Technik, die einzelne Element auf einer Fläche gesteuert hoch und herunterfahren kann. Diese Technik ist derzeit in der Entwicklung. Wie die Farbtiefe bei einem Bildschirm kann die Höhe der einzelnen Punkte eingestellt werden. \\
Mit 8 Höheneinstellungen und somit 3 Bit Höhentiefe benötigt ein Feld mit 20 auf 20 einzelnen Punkten einen Speicher in der Größe von $ 61,44\,\mathrm{MByte}$.
Eine weitere technische Weiterentwicklung kann Animationen auf der Mittelkonsole abspielen. Die Datenmenge steigert sich dementsprechend mit der Bildwiederholungsrate und der Dauer der Animation. Bei einer Bildwiederholungsrate von $ 4\,\mathrm{fps} $ und 20 Sekunden Dauer beträgt der benötigte Speicher $ 61,44\,\mathrm{MByte}$.
\subsection{Durchsichtiger Bildschirm im Dachfenster}
Rechtlich betrachtet ist die 
Technische Herausforderungen sind die Temperaturentwicklung durch Sonneneinstrahlung und die Kontaktierung bei Bewegung des Glasdaches.
Das Display hat eine ungefähre Leistung von $ XX\,\mathrm{W} $. 
Ein Standbild benötigt bei einer Pixelanzahl pro Zeile von $ 1920 $ und von $ 1080 $ Zeilen ein Speichergröße von ca. $ 256\,\mathrm{kByte}$. 
\begin{align}
	S &= \frac{1920\,\frac{\mathrm{Pixel}}{\mathrm{Zeile}}\times 1080\,\mathrm{Zeilen} \times 24\,\frac{\mathrm{Bit}}{\mathrm{Pixel}}}{12} \\
	&= 2.048.000\,\mathrm{Bit} = 2,048\,\mathrm{MBit} = 256\,\mathrm{KByte}
\end{align}
Für eine Animation von zehn Sekunden Länge benötigt das Display einen Speicher von $ 61,44\,\mathrm{MByte}$.
Insgesamt bei 50 Videos beträgt der benötigte Speicher $ 61,44\,\mathrm{MByte}$.
\subsection{LED-Matrix im Dachhimmel}
Herausforderungen:
Absicherung/ Crash Test, ideale Verbauposition
Hitzeentwicklung auf Dach
Ansteuerung, Montage/ Konstruktion
Lichtleiterkonzepte
Die LED-Matrix hat eine ungefähre Leistung von $ XX\,\mathrm{W} $. 
Ein Standbild benötigt bei einer Pixelanzahl pro Zeile von $ 192 $ und von $ 96 $ Zeilen ein Speichergröße von ca. $ 256\,\mathrm{kByte}$. 
\begin{align}
	S &= \frac{192\,\frac{\mathrm{Pixel}}{\mathrm{Zeile}}\times 96\,\mathrm{Zeilen} \times 24\,\frac{\mathrm{Bit}}{\mathrm{Pixel}}}{12} \\
	&= 2.048.000\,\mathrm{Bit} = 2,048\,\mathrm{MBit} = 256\,\mathrm{KByte}
\end{align}
Für eine Animation von zehn Sekunden Länge benötigt das Display einen Speicher von $ 61,44\,\mathrm{MByte}$.
Insgesamt bei 50 Videos beträgt der benötigte Speicher $ 61,44\,\mathrm{MByte}$.
\subsection{Duftflakons im Innenraum}
Eine technische Verbreitung unterschiedlicher individueller Düfte ist in diesem Konzept nicht realisiert. Ein aktueller technischer Aufbau besteht aus einem Set mit unterschiedlichen Duftflakons, die einzeln geöffnet werden. Dadurch ist die Anzahl der Düfte beschränkt auf die Set-Größe.
\subsection{Bildschirmoberflächen im Cockpit}
Das Anzeigen individualisierter Bildschirmoberflächen auf vorhandenen Displays ist unter den oben beschrieben Kriterien schon im Fahrzeug möglich und benötigt keine weitere Analyse.
\subsection{Soundplayer im Innenraum}
Das Abspielen individualisierter Sounds durch vorhandene Lautsprecher ist unter den oben beschrieben Kriterien schon im Fahrzeug möglich und benötigt keine weitere Analyse.
\subsection{Gesamt-technische Zusammenfassung}
In der Tabelle \ref{tab:Speicherbedarf} werden die Komponenten, die visuelle oder haptische Anzeigen sind, mit dem benötigten Speicher für die Anzeige eines Bildes dargestellt. Die Berechnung erfolgt mit der Gleichung \ref{eq:Bilddatenmenge}.
\begin{table}[hbt]	
	\centering
	\renewcommand{\arraystretch}{1.5}	% Skaliert die Zeilenhöhe der Tabelle
	\captionabove[Berechnung des Speicherbedarfes für ein Bild]{Berechnung des Speicherbedarfes für ein Bild}
	\label{tab:Speicherbedarf}
	\begin{tabular}{c|cccc}
		\textbf{Komponente} & \parbox[t]{0.16\linewidth}{\centering Pixel} & \parbox[t]{0.16\linewidth}{\centering Bitanzahl \\pro Pixel} & \parbox[t]{0.16\linewidth}{\centering Kompressions-\\faktor} & \parbox[t]{0.16\linewidth}{\centering Speicher} \\ 
		\hline 
		\hline 
		\parbox[t]{0.4\linewidth}{\centering E-Papier in der Frontschürze} & $ 2560 \times 1440 $ & $ 8 $ & $ 12 $ & $ 307,2\,\mathrm{kByte} $\\ \parbox[t]{0.4\linewidth}{\centering E-Papier Embleme über\\den vorderen Radkästen} & $ 1600 \times 1200 $ & $ 8 $ & 12 & $ 160\,\mathrm{kByte} $ \\
		\parbox[t]{0.4\linewidth}{\centering E-Papier in der Heckleuchte} & $ 1600 \times 1200 $ & $ 8 $ & $ 12 $ & $ 160\,\mathrm{kByte} $ \\
		\parbox[t]{0.4\linewidth}{\centering LED-Streifen in der Frontschürze} & $ 332 \times 1 $ & $ 24 $ & $ 12 $ & $ 83\,\mathrm{Byte} $ \\
		\parbox[t]{0.4\linewidth}{\centering LED-Streifen in den Radkästen} & $ 200 \times 1 $ & $ 24 $ & $ 12 $ & $ 50,\mathrm{Byte} $\\ \parbox[t]{0.4\linewidth}{\centering LED-Streifen in der Heckleuchte} & $ 391 \times 1 $ & $ 24 $ & $ 12 $ & $ 98,\mathrm{Byte} $ \\ 
		\parbox[t]{0.4\linewidth}{\centering LED-Streifen im Interieur} & $ 391 \times 1 $ & $ 24 $ & $ 12 $ & $ X\,\mathrm{Byte} $ \\
		\parbox[t]{0.4\linewidth}{\centering LED Türtafeln} & $ 4 \times 1 $ & $ 24 $ & $ 12 $ &  $ X\,\mathrm{Byte} $ \\
		\parbox[t]{0.4\linewidth}{\centering Videoprojektoren in\\den Außenspiegeln} & $ 1280 \times 800 $ & $ 24 $ & $ 12 $ &   $ X\,\mathrm{Byte} $ \\ 
		\parbox[t]{0.4\linewidth}{\centering Videoprojektoren im Fußraum} & $ 1280 \times 800 $ & $ 24 $ & $ 12 $ & $ X\,\mathrm{Byte} $ \\
		\parbox[t]{0.4\linewidth}{\centering Bildschirme in den\\hinteren Seitenfenstern} & $ 1280 \times 800 $ & $ 24 $ & $ 12 $ & $ X\,\mathrm{Byte} $ \\
		\parbox[t]{0.4\linewidth}{\centering Bildschirme in der Einstiegsleiste} &  $ 1280 \times 1024 $ & $ 24 $ & $ 12 $ & $ X\,\mathrm{Byte} $ \\
		\parbox[t]{0.4\linewidth}{\centering Durchsichtiger Bildschirm\\im Dachfenster} & $ 1920 \times 1080 $ & $ 24 $ & $ 12 $ & $ X\,\mathrm{Byte} $ \\
		LED-Matrix im Dachhimmel & $ 192 \times 96 $ & $ 24 $ & $ 12 $ & $ X\,\mathrm{Byte} $ \\
		\parbox[t]{0.4\linewidth}{\centering Morphende Oberfläche\\in der Mittelkonsole} & $ X \times X $ & $ 8 $ & $ 1 $ & $ X\,\mathrm{Byte} $ \\
		\parbox[t]{0.4\linewidth}{\centering Bildschirmoberflächen im Cockpit} & $ X \times X $ & $ 24 $ & $ 12 $ & $ X\,\mathrm{Byte} $ \\
	\end{tabular} 
\end{table}
\begin{table}[hbt]	
	\centering
	\renewcommand{\arraystretch}{1.5}	% Skaliert die Zeilenhöhe der Tabelle
	\captionabove[Liste der Komponenten mit den oben festgestellten Werten]{Liste der Komponenten mit den oben festgestellten Werte}
	\label{tab:Werte}
	\begin{tabular}{c|ccc}
		\textbf{Komponente} & \parbox[t]{0.16\linewidth}{\centering Leistungs-\\bedarf} & \parbox[t]{0.16\linewidth}{\centering Speicher-\\größe} & \parbox[t]{0.16\linewidth}{\centering Echtzeit-\\anbindung} \\ 
		\hline 
		\hline 
		\parbox[t]{0.4\linewidth}{\centering E-Papier in der Frontschürze} & $ X\,\mathrm{W} $ & $ X\,\mathrm{Byte} $ & X \\ \parbox[t]{0.4\linewidth}{\centering E-Papier Embleme über\\den vorderen Radkästen} & $ X\,\mathrm{W} $ & $ X\,\mathrm{Byte} $ & X \\\parbox[t]{0.4\linewidth}{\centering E-Papier in der Heckleuchte} & $ X\,\mathrm{W} $ & $ X\,\mathrm{Byte} $ & X \\
		\parbox[t]{0.4\linewidth}{\centering LED-Streifen in der Frontschürze} & $ X\,\mathrm{W} $ & $ X\,\mathrm{Byte} $ & X \\
		\parbox[t]{0.4\linewidth}{\centering LED-Streifen in den Radkästen} & $ X\,\mathrm{W} $ & $ X\,\mathrm{Byte} $ & X \\ \parbox[t]{0.4\linewidth}{\centering LED-Streifen in der Heckleuchte} & $ X\,\mathrm{W} $ & $ X\,\mathrm{Byte} $ & X \\ \parbox[t]{0.4\linewidth}{\centering LED-Streifen im Interieur} & $ X\,\mathrm{W} $ & $ X\,\mathrm{Byte} $ & X \\
		\parbox[t]{0.4\linewidth}{\centering LED Türtafeln} & $ X\,\mathrm{W} $ & $ X\,\mathrm{Byte} $ & X \\
		\parbox[t]{0.4\linewidth}{\centering Videoprojektoren in\\den Außenspiegeln} & $ X\,\mathrm{W} $ & $ X\,\mathrm{Byte} $ & X \\ \parbox[t]{0.4\linewidth}{\centering Videoprojektoren im Fußraum} & $ X\,\mathrm{W} $ & $ X\,\mathrm{Byte} $ & X \\
		\parbox[t]{0.4\linewidth}{\centering Bildschirme in den\\hinteren Seitenfenstern} &$ X\,\mathrm{W} $ & $ X\,\mathrm{Byte} $ & X \\
		\parbox[t]{0.4\linewidth}{\centering Bildschirme in der Einstiegsleiste} & $ X\,\mathrm{W} $ & $ X\,\mathrm{Byte} $ & X \\
		\parbox[t]{0.4\linewidth}{\centering Durchsichtiger Bildschirm\\im Dachfenster} & $ X\,\mathrm{W} $ & $ X\,\mathrm{Byte} $ & X \\
		LED-Matrix im Dachhimmel & $ X\,\mathrm{W} $ & $ X\,\mathrm{Byte} $ & X \\
		\parbox[t]{0.4\linewidth}{\centering Morphende Oberfläche\\in der Mittelkonsole} & $ X\,\mathrm{W} $ & $ X\,\mathrm{Byte} $ & X \\
		\parbox[t]{0.4\linewidth}{\centering Bildschirmoberflächen im Cockpit} & $ X\,\mathrm{W} $ & $ X\,\mathrm{Byte} $ & X \\
		\parbox[t]{0.4\linewidth}{\centering Soundplayer im Innenraum} & $ X\,\mathrm{W} $ & $ X\,\mathrm{Byte} $ & X \\
		\parbox[t]{0.4\linewidth}{\centering Duftflakons im Innenraum} & $ X\,\mathrm{W} $ & $ X\,\mathrm{Byte} $ & X \\
	\end{tabular} 
\end{table}
\section{Gesamtwirtschaftliche Analyse}
In diesem Abschnitt werden die Komponenten als Gesamtes unter den wirtschaftlichen Kriterien betrachtet, weil die Komponenten nicht als einzelne Sonderausstattung, sondern als Ganzes oder Gebündelt vertrieben werden soll. Daher ist eine Gesamtbetrachtung der wirtschaftlichen Kriterien sinnvoll. \\
In Tabelle \ref{tab:Entwicklung} werden alle Komponenten mit den nötigen Anpassungen in der Serienentwicklung in den bestimmten Bereichen auf Basis der vorherigen technischen Analyse. 
\begin{table}[hbt]	
	\centering
	\renewcommand{\arraystretch}{1.5}	% Skaliert die Zeilenhöhe der Tabelle
	\captionabove[Liste der Komponenten mit den nötigen Eingriffen in die Fahrzeugentwicklung]{Liste der Komponenten mit den nötigen Eingriffen in die Fahrzeugentwicklung}
	\label{tab:Entwicklung}
	\begin{tabular}{c|ccc}
		\textbf{Komponente} & \textbf{Karosserie} & \textbf{E/E Architektur} & \textbf{Software} \\ 
		\hline 
		\hline 
		\parbox[t]{0.4\linewidth}{\centering E-Papier in der Frontschürze} & X & X & X \\
		\parbox[t]{0.4\linewidth}{\centering LED-Streifen in der Frontschürze} & X & X & X \\
		\parbox[t]{0.4\linewidth}{\centering E-Papier Embleme über\\den vorderen Radkästen} & X & X & X \\
		\parbox[t]{0.4\linewidth}{\centering LED-Streifen in den Radkästen} & X & X & X \\
		\parbox[t]{0.4\linewidth}{\centering Videoprojektoren in\\den Außenspiegeln} & X & X & X \\
		\parbox[t]{0.4\linewidth}{\centering Bildschirme in den\\hinteren Seitenfenstern} & X & X & X \\
		\parbox[t]{0.4\linewidth}{\centering LED-Streifen in der Heckleuchte} &  & X & X \\
		\parbox[t]{0.4\linewidth}{\centering E-Papier in der Heckleuchte} & X & X & X \\
		\parbox[t]{0.4\linewidth}{\centering LED-Streifen im Interieur} &  & X & X \\
		\parbox[t]{0.4\linewidth}{\centering LED Türtafeln} &  & X & X \\
		\parbox[t]{0.4\linewidth}{\centering Bildschirme in der Einstiegsleiste} &  & X & X \\
		\parbox[t]{0.4\linewidth}{\centering Videoprojektoren im Fußraum} &  & X & X \\
		\parbox[t]{0.4\linewidth}{\centering Morphende Oberfläche\\in der Mittelkonsole} &  & X & X \\
		\parbox[t]{0.4\linewidth}{\centering Durchsichtiger Bildschirm\\im Dachfenster} &  & X & X \\
		LED-Matrix im Dachhimmel & X & X & X \\
		\parbox[t]{0.4\linewidth}{\centering Bildschirmoberflächen im Cockpit} &  &  & X \\
		\parbox[t]{0.4\linewidth}{\centering Soundplayer im Innenraum} &  & X & X \\
		\parbox[t]{0.4\linewidth}{\centering Duftflakons im Innenraum} &  & X & X \\
	\end{tabular} 
\end{table}
		%Der parbox-Befehl ist erforderlich, damit ein Zeilenumbruch erzeugt werden kann. c-Spalten (zentriert) erlauben nicht automatisch einen Zeilenumpruch. Linksbündig gesetzte p-Spalten erlauben automatisch den Zeilenumbruch.%

