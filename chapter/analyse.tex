\chapter{Analyse der Komponenten des Prototypen}
\label{cha:Analyse}
Nachfolgend werden alle Komponenten unter den von \ref{cha:Kriterien} genannten Kriterien analysiert. Im Zuge der Analyse werden im nächsten Kapitel Entwürfe für die Implementierung der Komponenten dargestellt. Die Analyse basiert auf den Komponenten des Prototypen, aber beschränkt sich nicht auf die einzeln verbaute, sonder geht auf den Gesamtkontext der Komponente ein.
Die Kriterien für den Einbau und die spätere Analyse basieren zum Einen auf einer selbstständigen Recherche über die Komponenten und den relevanten Bedingungen. Zum Anderen wurden Informationen von Experten in den einzelnen Fachgebieten der Karosserieinnenentwicklung, Individualisierung, After-Sales-Produktentwicklung herangezogen.
\section{Kriterien für den Einbau der Komponenten in Serienfahrzeuge}
\label{cha:Kriterien}
Die folgenden Kriterien richten sich zum Einen an die Komponenten und zum Anderen an die Fahrzeugentwicklung. Anhand der Kriterien kann eine Analyse des Ist-Zustandes angewendet werden.\\
Da für den Verkauf und Betrieb von Fahrzeugen die gültigen Gesetze und Normen eingehalten werden müssen, haben die rechtlichen Kriterien die oberste Priorität. Wenn alle rechtlichen Bedingungen erfüllt sind, stellen sich betriebswirtschaftliche Fragestellungen, ob die Komponenten in Ihrer Funktion einen Nutzen für den Kunden und für das Unternehmen haben. Dies können hier besonders auch optische Vorteile sein. Zuletzt müssen technische Kriterien von der Seite der Fahrzeugentwicklung und der Komponenten erfüllt sein, damit die Weiterentwicklung der Komponenten in Erwägung gezogen werden kann. \\
Durch den zeitlichen Rahmen dieser Arbeit und dem fachlichen Schwerpunkt in elektrotechnischen Systemen, liegt der Fokus der Analyse in der technischen Analyse des Strombedarfs und der Versorgung mit Daten. Dennoch soll diese Arbeit die wichtigsten Aspekte und eine Einordnung der anderen Fachbereiche liefern.\\
Nachfolgend werden mit der gleichen Reihenfolge wie oben die Kriterien definiert.
\subsection{Rechtliche Kriterien}
Unter rechtlichen Kriterien finden sich alle Bedingungen für die Fahrzeugentwicklung, die auf Basis von Gesetzen und Normen für eine Zulassung erfüllt werden müssen. Diese Kriterien gewährleisten zugleich Rechtssicherheit für den Hersteller. \\
Aus Gründen des Überblicks wird in der folgenden Arbeit nur auf einzelne Themen von rechtlichen Kriterien eingegangen. Für Lichter am Fahrzeug gilt die europäische Richtlinie ECE R48. Im Innenraum muss die Vermeidung von Fahrerablenkung sichergestellt werden.\\
Im Rahmen einer weiteren Entwicklung von Komponenten für eine Fahrzeugserie sind die rechtlichen Bedingungen unbedingt frühzeitig von internen oder externen Juristen und Juristinnen zu prüfen.
\subsection{Wirtschaftliche Kriterien}
Unter wirtschaftliche Kriterien fallen sowohl betriebswirtschaftliche Betrachtungen als auch kundenspezifische Anforderungen. Diese Kriterien sind Grundvoraussetzungen für eine Erstellung eines Geschäftsplan und sollen somit einen Rahmen in diesem Bereich bieten. Die weiteren Schritte nach Erfüllung dieser Kriterien sind Absprachen mit Produktstrategen, Marketingexperten und Kalkulatoren. \\
Unter dem betriebswirtschaftlichen Kriterium steht in erster Linie die Gewinnerzielungsabsicht, das heißt die Kosten des Produkts sollen geringer als die möglichen Einnahmen sein. Kosten können neben den Entwicklungskosten und Produktionskosten für die Komponente, auch Verwaltungskosten über den Lebenszyklus des Produkts und Anpassungskosten anderer Bauteile im Fahrzeug sein. \\
Einnahmen sind neben dem Verkauf der Komponente als Sonderausstattung weitere sekundäre Möglichkeiten durch digitale Produkte spezifisch zu den Komponenten. Daneben können es auch nicht monetäre Einnahmen geben, die das gesamte Produkt und die Marke in ihrem Ansehen stärken.\\
Um einen Gewinn zu erwirtschaften, muss die Komponente mit ihren Fixkosten wie die Entwicklung und ihren variablen Kosten wie den Produktionskosten gering sein, aber dennoch einen großen Mehrwert für den Kunden liefern, damit die Nachfrage an der Komponente steigt. \\
Die Entwicklungskosten sind abhängig vom aktuellen Reifegrad vorhandener Technik und der Komplexität der Technik, während Anpassungskosten abhängig sind von dem entstehenden Bedarf an Architekturveränderungen. Die Produktionskosten sind abhängig von den Einkaufspreisen der Komponenten und der Montage. Diese Kriterien werden in dieser Arbeit nach der technischen Analyse gesammelt betrachtet, um einen Vergleich mit den einzelnen Komponenten zu liefern. \\

Kundenspezifische Anforderungen sind:
\begin{itemize}
	\item Die Komponente muss einen sichtbaren Mehrwert für den Kunden bieten.
	\item Die Komponente darf andere Funktionen nicht beeinträchtigen.
\end{itemize}
\subsection{Technische Kriterien}
Für den Einbau der Komponente müssen folgende technische Kriterien erfüllt sein. Das Fahrzeug muss genügend Bauraum für die Komponenten zur Verfügung stellen, ausreichende Stromversorgung für den Betrieb der Komponenten, genügend Bandbreite für den Informationsaustausch.
\paragraph{Einbau}
Unter Einbau fallen alle Kriterien, die erfüllt sein müssen, damit die Komponenten sachgerecht in ein Fahrzeug integriert werden können.
Das Fahrzeug muss an der gewünschten Stelle den benötigten Bauraum für die Komponenten bereitstellen. Daneben an der Stelle Möglichkeiten für die Befestigung der Komponenten vorhanden sein. Des weiteren ist das Gewicht der Komponenten auf Verträglichkeit zur Fahrzeugstruktur zu prüfen. Daneben müssen die Komponenten Robust und langlebig, im Besonderen Crash-technisch untersucht, sein.

\paragraph{Versorgung}
Unter Versorgung steht zum Einen die Versorgung mit Elektrischen Strom und zum Anderen die Versorgung mit Informationen.\\
Die Versorgung mit Strom erfolgt über das Niedervolt Bordnetz des Fahrzeugs, je nach Komponente müssen die benötigten Spannungen an das Bordnetz angepasst werden.\\
Die Versorgung mit Informationen geschieht entweder über drahtlose oder kabelgebundene Bussysteme.
Je nach Bedarf an Bandbreite mu
\paragraph{Rechenleistung}

\section{Analyse der Exterieur Komponenten}
\subsection{E-Papier in der Frontschürze}
\subsection{LED-Streifen in der Frontschürze}
\subsection{E-Papier Embleme über den vorderen Radkästen}
\subsection{LED-Streifen in den Radkästen}
\subsection{Videoprojektoren in den Außenspiegeln}
\subsection{Bildschirme in den hinteren Seitenfenstern}
\subsection{LED-Streifen in der Heckleuchte}
\subsection{E-Papier in der Heckleuchte}
\section{Analyse der Interieur Komponenten}
\subsection{LED-Streifen im Interieur}
\subsection{Matrix LED Türtafeln}
\subsection{Bildschirme in der Einstiegsleiste}
\subsection{Videoprojektoren im Fußraum}
\subsection{Morphende Oberfläche in der Mittelkonsole}
\subsection{Durchsichtiger Bildschirm im Dachfenster}
\subsection{LED Matrix im Dachhimmel}
\subsection{Duftflakons im Innenraum}
\subsection{Bildschirmoberflächen im Cockpit}
\subsection{Soundplayer im Innenraum}
\section{Gesamtwirtschaftliche Analyse}
In Tabelle \ref{tab:Entwicklung} werden alle Komponenten mit den nötigen Anpassungen in der Serienentwicklung in den bestimmten Bereichen auf Basis der vorherigen technischen Analyse. 
\begin{table}[hbt]	
	\centering
	\renewcommand{\arraystretch}{1.5}	% Skaliert die Zeilenhöhe der Tabelle
	\captionabove[Liste der Komponenten mit den nötigen Eingriffen in die Fahrzeugentwicklung]{Liste der Komponenten mit den nötigen Eingriffen in die Fahrzeugentwicklung}
	\label{tab:Entwicklung}
	\begin{tabular}{c|ccc}
		\textbf{Komponente} & \textbf{Karosserie} & \textbf{E/E Architektur} & \textbf{Software} \\ 
		\hline 
		\hline 
		E-Papier in der Frontschürze & X & X & X \\
		LED-Streifen in der Frontschürze & X & X & X \\
		E-Papier Embleme über den vorderen Radkästen & X & X & X \\
		LED-Streifen in den Radkästen & X & X & X \\
		Videoprojektoren in den Außenspiegeln & X & X & X \\
		Bildschirme in den hinteren Seitenfenstern & X & X & X \\
		LED-Streifen in der Heckleuchte &  & X & X \\
		E-Papier in der Heckleuchte & X & X & X \\
		LED-Streifen im Interieur &  & X & X \\
		Matrix LED Türtafeln &  & X & X \\
		Bildschirme in der Einstiegsleiste &  & X & X \\
		Videoprojektoren im Fußraum &  & X & X \\
		Morphende Oberfläche in der Mittelkonsole &  & X & X \\
		Durchsichtiger Bildschirm im Dachfenster &  & X & X \\
		LED Matrix im Dachhimmel & X & X & X \\
		Bildschirmoberflächen im Cockpit &  &  & X \\
		Soundplayer im Innenraum &  & X & X \\
		Duftflakons im Innenraum &  & X & X \\
	\end{tabular} 
\end{table}
		%Der parbox-Befehl ist erforderlich, damit ein Zeilenumbruch erzeugt werden kann. c-Spalten (zentriert) erlauben nicht automatisch einen Zeilenumpruch. Linksbündig gesetzte p-Spalten erlauben automatisch den Zeilenumbruch.%

