\chapter{Analyse der Komponenten des Prototypen}
\label{cha:Analyse}
Nachfolgend werden alle Komponenten unter den von \ref{cha:Kriterien} genannten Kriterien analysiert. Im Zuge der Analyse werden im nächsten Kapitel Entwürfe für die Implementierung der Komponenten dargestellt. Die Analyse basiert auf den Komponenten des Prototypen, aber beschränkt sich nicht auf die einzeln verbaute, sonder geht auf den Gesamtkontext der Komponente ein.
Die Kriterien für den Einbau und die spätere Analyse basieren zum Einen auf einer selbstständigen Recherche über die Komponenten und den relevanten Bedingungen. Zum Anderen wurden Informationen von Experten in den einzelnen Fachgebieten der Karosserieinnenentwicklung, Marktforschung, After-Sales-Produktentwicklung herangezogen.
\section{Kriterien für den Einbau der Komponenten in Serienfahrzeuge}
\label{cha:Kriterien}
Kriterien sind im folgenden Bedingungen, die erfüllt sein müssen, um die im Prototypen beschriebenen Komponenten in eine Serienentwicklung einzubringen.\\
Da für den Verkauf und Betrieb von Fahrzeugen die gültigen Gesetze und Normen eingehalten werden müssen, haben die rechtlichen Kriterien die oberste Priorität. Wenn alle rechtlichen Bedingungen erfüllt sind, stellt sich betriebswirtschaftliche Fragestellungen, ob die Komponenten in Ihrer Funktion einen Nutzen für den Kunden und für das Unternehmen haben. Dies können hier besonders auch optische Nutzen sein. Zuletzt müssen technische Kriterien von der Fahrzeug und der Komponenten Seite erfüllt sein, damit die Weiterentwicklung der Komponenten in Erwägung gezogen werden kann.
\subsection{Rechtliche Kriterien}
Nachfolgend werden die wichtigsten rechtlichen Kriterien für die Komponenten erläutert. Um den Rahmen dieser Arbeit nicht zu überziehen, wird im Folgenden von der Zulassung von Fahrzeugen in Deutschland ausgegangen.
Für Lichter gilt die ECE R48
\subsection{Wirtschaftliche Kriterien}

\subsection{Technische Kriterien}
Für den Einbau der Komponente müssen folgende technische Kriterien erfüllt sein. Das Fahrzeug muss genügend Bauraum für die Komponenten zur Verfügung stellen, ausreichende Stromversorgung für den Betrieb der Komponenten, genügend Bandbreite für den Informationsaustausch.
\subsubsection{Verbau}
\subsubsection{Versorgung}
\subsubsection{Anpassungen}

\section{Analyse der Exterieur Komponenten}
\subsection{E-Papier in der Frontschürze}

\subsection{LED-Streifen in der Frontschürze}

\subsection{E-Papier Embleme über den vorderen Radkästen}

\subsection{LED-Streifen in den Radkästen}
 
\subsection{Videoprojektoren in den Außenspiegeln}
\subsection{Bildschirme in den hinteren Seitenfenstern}
\subsection{LED-Streifen in der Heckleuchte}
\subsection{E-Papier in der Heckleuchte}


\section{Analyse der Interieur Komponenten}
\subsection{LED-Streifen im Interieur}
\subsection{Matrix LED Türtafeln}
\subsection{Bildschirme in der Einstiegsleiste}
\subsection{Videoprojektoren im Fußraum}
\subsection{Morphende Oberfläche in der Mittelkonsole}
\subsection{Durchsichtiger Bildschirm im Dachfenster}
\subsection{LED Matrix im Dachhimmel}