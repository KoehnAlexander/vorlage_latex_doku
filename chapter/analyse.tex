\chapter{Anforderungsanalyse der Komponenten des Prototypen für eine Serienimplementierung}
\label{cha:Analyse}
Nachfolgend werden alle Komponenten unter den von \ref{cha:Kriterien} genannten Kriterien analysiert. Im Zuge der Analyse werden im nächsten Kapitel Entwürfe für die informationstechnische Anbindung der Komponenten dargestellt.\\
Die Analyse basiert auf den Komponenten des Prototypen, aber beschränkt sich nicht auf einzeln verbaute Technologien, sondern geht auf den Gesamtkontext der Komponenten ein. Grundlegend richtet sich die Prüfung nach dem sinnlichen Darstellungszweck und dessen Bedarf an technischen Eigenschaften. \\
Am Beispiel des Bildschirms in der Einstiegsleiste wird die Größe im Prototyp in Bezug auf seine Funktionalität hinsichtlich der Darstellung der vorgesehenen Inhalte geprüft. \\
Die Kriterien für den Einbau und die spätere Analyse basieren zum einen auf einer selbstständigen Recherche über die Komponenten und den relevanten Bedingungen. Zum anderen wurden Informationen von Experten in den einzelnen Fachgebieten der Karosserieinnenentwicklung, Individualisierung, After-Sales-Produktentwicklung herangezogen, um die Eignung nach rechtlichen, wirtschaftlichen und technischen Kriterien zu bewerten. \\
Für einen Überblick über die Erkenntnisse fasst das Unterkapitel \ref{ZusammendassungAnalyse} die Analyse zusammen.
\section{Kriterien für den Einbau der Komponenten in Serienfahrzeuge}
\label{cha:Kriterien}
Mit den folgenden Kriterien kann eine Analyse des Ist-Zustandes der Komponenten durchgeführt werden.
Da für den Verkauf und Betrieb von Fahrzeugen die gültigen Gesetze und Normen eingehalten werden müssen, haben die rechtlichen Kriterien die oberste Priorität in der Beurteilung der Machbarkeit. Wenn alle rechtlichen Bedingungen erfüllt sind, stellen sich betriebswirtschaftliche Fragestellungen, ob die Komponenten in ihrer Funktion einen Nutzen für den Kunden und für das Unternehmen haben. Zuletzt müssen technische Kriterien von der Seite der Fahrzeugentwicklung und der Komponenten erfüllt sein, damit die Weiterentwicklung der Komponenten in Erwägung gezogen werden kann. \\
Durch den zeitlichen Rahmen dieser Arbeit und dem fachlichen Schwerpunkt in elektrotechnischen Systemen liegt der Fokus in der technischen Analyse des Informationsbedarfs. Zusätzlich soll diese Arbeit die wichtigsten Aspekte und eine Einordnung der anderen Fachbereiche liefern.\\
Nachfolgend werden mit der gleichen Reihenfolge wie oben die Kriterien definiert.
\subsection{Rechtliche Kriterien}
Unter rechtlichen Kriterien finden sich alle Vorgaben für die Fahrzeugentwicklung, die auf Basis von Gesetzen und Normen für eine Zulassung erfüllt werden müssen. Diese Kriterien gewährleisten zugleich Rechtssicherheit für den Hersteller. \\
Aus Gründen des Überblicks wird in der folgenden Arbeit nur auf einzelne Themen von rechtlichen Kriterien eingegangen. 
Ein Beispiel für rechtliche Kriterien ist die europäische Richtlinie ECE R48 für Lichter am Fahrzeug. \cite[Vgl. Seite 1 ff.]{R48.2016} \\
Im Rahmen einer weiteren Entwicklung von Komponenten für eine Fahrzeugserie sind die rechtlichen Bedingungen unbedingt frühzeitig von internen oder externen Juristen und Juristinnen zu prüfen.
\subsection{Wirtschaftliche Kriterien}
Unter wirtschaftliche Kriterien fallen sowohl betriebswirtschaftliche Betrachtungen als auch kundenspezifische Anforderungen. Diese Aspekte sind Grundvoraussetzung für die Erstellung eines positiven Geschäftsplans, in Abstimmung mit Produktstrategen, Marketingexperten und Kalkulatoren. \\
In einer ersten Produktbewertung können alle möglichen Leistungen zu den Kosten verglichen werden.
\paragraph{Leistungen}
Zu den Leistungen gehören in erster Linie zusätzliche Umsatzerlöse durch den Verkauf der Komponenten als Sonderausstattung und der digitalen Inhalte. Darüber hinaus werden nicht monetäre Leistungen durch Imageverbesserung erzielt.
\paragraph{Kosten}
Kosten entstehen durch Entwicklung, Beschaffung, Logistik Produktion, Verwaltung und Marketing. 
\paragraph{Gesamtbetrachtung}
In der folgenden Tabelle \ref{tab:Kosten} werden die möglichen Einnahmen den Kosten gegenüber gestellt.
\begin{table}[hbt]	
	\centering
	\renewcommand{\arraystretch}{1.5}	% Skaliert die Zeilenhöhe der Tabelle
	\captionabove[Mögliche Einnahme- und Kostenquellen]{Mögliche Einnahme- und Kostenquellen}
	\label{tab:Kosten}
	\begin{tabular}{l|l}
		\textbf{Kosten} & \textbf{Leistungen} \\ 
		\hline
		\hline
		Entwicklung & Verkauf als Sonderausstattung \\
		Beschaffung &  Verkauf von Kollektionen \\
		Logistik & Verkauf von digitalen Diensten \\
		Produktion & Aufwertung des Markenimages \\
		Verwaltung & \\
		Marketing & \\
	\end{tabular} 
\end{table}
Diese wirtschaftlichen Kriterien werden in dieser Arbeit nach der technischen Analyse gesammelt betrachtet. Das Gesamtkonzept für das Fahrzeug fordert eine ganzheitliche Inszenierung des Fahrzeugs, wofür mehrere Komponenten benötigt werden, um diesen Effekt zu erzielen. Aus diesem Hintergrund ist es sinnvoll die Entwicklung der Komponenten ganzheitlich wirtschaftlich zu bewerten. \\
\subsection{Technische Kriterien}
Unter technischen Kriterien fallen alle relevanten Gebiete für eine erste Untersuchung für die Produktentwicklung:
\begin{enumerate}
	\item Physikalische Dimensionierung
	\item Stabilität
	\item Gewicht
	\item Elektrischer Energiebedarf
	\item Optik
	\item Wartungsfähigkeit
	\item Informationsbedarf
\end{enumerate}
Je nach Beschaffenheit der Komponenten können diese Gebiete unterschiedlich relevant sein.
\paragraph{Physikalische Dimensionierung}
Zur erfolgreichen Integration der Komponente in ein Fahrzeug muss an der gewünschten Stelle genügend Bauraum zur Verfügung stehen.
\paragraph{Stabilität}
Die Komponente muss an der Einbaustelle befestigt werden können. Alle Stabilität- und Crashtests müssen positiv ausfallen.
\paragraph{Gewicht}
Das Fahrzeug muss das zusätzliche Gewicht der Komponenten an dieser Stelle aufnehmen können.
\paragraph{Elektrischer Energiebedarf}
Die Komponente soll einen möglichst geringen Strombedarf haben, die das Fahrzeug hinreichend zur Verfügung stellen kann.
\paragraph{Optische Kriterien}
Unter optischen Kriterien fallen alle allgemein gültigen Designprinzipien in der Produktentstehung. Darunter versteht sich die Harmonie der Komponenten mit der Umgebung, die Vermeidung von Bildschirmrändern usw.. \\
Daneben ist die Einhaltung einer genügend hohen Pixeldichte nach der Formel aus Kapitel \ref{cha:Grundlagen} zu gewährleisten.
\paragraph{Wartungsfähigkeit}
Unter Wartungsfähigkeit fällt der leichte Ein- und Ausbau der Komponente bei Bedarf und der Austausch von Verschleißteile.
\paragraph{Informationsbedarf}
Zur Steuerung der Komponente muss eine Möglichkeit vorhanden sein, mit dem Gesamtfahrzeug zu kommunizieren. Kriterien sind zum einen die benötigte Bandbreite der Komponenten und die Geschwindigkeit des Informationsaustausches. \\
Die Berechnung des Speicherbedarfes dient als erste Abschätzung für die Speicherdimensionierung und kann vom tatsächlichen Bedarf abweichen, da die Rechnung Prüf- und Steuerdaten nicht beachtet.
\paragraph{Berechnungsannahmen}
Für die Berechnung des Informationsbedarfs und des Speicherbedarfs müssen Annahmen getroffen werden, da in dieser Projektphase diese noch nicht getroffen wurden. \\
Als erste Annahme sind genau zehn Kollektionen im Fahrzeug gespeichert. Jede Kollektion hat aber nicht nur einen Inhalt pro Komponente, sondern möglicherweise angepasste Inhalte für unterschiedliche Trigger. Deshalb soll für die Speicherabschätzung mit fünf Inhalten pro Komponente ausgegangen werden.\\
Inhalte können entweder Bilder oder Videos sein.
Bei Bilder und Videos wird ein Kompressionsfaktor von zehn genutzt, da dieser Wert ein üblicher Mittelwert in der Kompression zum Beispiel mit JPEG ist. Bei Anzeigeflächen mit geringer Pixelanzahl wird ohne Kompressionsfaktor gerechnet, da dort die Datenmengen von Vornherein gering sind. \\
Die Farbtiefe wird mit 8 Bit pro Farbe angesetzt, da dies zum Beispiel bei JPEG genutzt wird.
Bei Videos wird mit einer Frame-Rate von 24 Bilder pro Sekunde ausgegangen und einer Länge von zehn Sekunden.
\section{Analyse der Exterieur-Komponenten}
Im folgenden werden die Exterieur-Komponenten an Hand der oben beschriebenen einzelnen Kriterien geprüft. Die wirtschaftlichen Kriterien werden in einem separaten Abschnitt gesammelt betrachtet. 
\subsection{E-Papier in der Frontschürze}
Rechtlich ist zu prüfen, ob das E-Papier als Leuchte gilt und danach behandelt werden muss. Daneben sind die rechtlichen Grundlagen für das Ändern der Bildschirminhalte in unterschiedlichen Fahrmodi, wie zum Beispiel Parken oder Fahren, zu analysieren. \\
Wirtschaftlich betrachtet ist diese Komponente ein zentraler Bestandteil des Fahrzeugkonzeptes, da durch die Größe und Lage der Kunde die Inhalte stark wahrnimmt. \\
Mit einer physikalischen Dimensionierung über die gesamte Breite zwischen den Frontlichtscheinwerfern und somit eine mögliche Breite bei unterschiedlichen Fahrzeugmodellen von $ 80 $ bis $ 100\,\mathrm{cm} $ und einer Höhe von $ 50 $ bis $ 60\,\mathrm{cm} $ ist die Stabilität des E-Papiers zu prüfen, da die Lage im Kotflügel besonders exponiert ist. \\ 
Das zusätzliche Gewicht des Bildschirms ist an der vorderen Kotflügelbefestigung aufzunehmen. \\
Mit einer Auflösung von 2560 Pixel pro Zeile zu 1440 Zeilen und einer Farbtiefe von 16 Stufen benötigt das Display mit der Formel \ref{eq:Bilddatenmenge} pro Bild einen Speicher S von $ 368,64\,\mathrm{kByte} $. 
\begin{align}
	S &= \frac{2560\,\frac{\mathrm{Pixel}}{\mathrm{Zeile}}\times 1440\,\mathrm{Zeilen} \times 8\,\frac{\mathrm{Bit}}{\mathrm{Pixel}}}{10} \\
	&=  368,64\,\mathrm{kByte}
\end{align}
Bei zehn Kollektionen mit jeweils fünf Bildern benötigt man einen Speicher von $ 18,432\,\mathrm{MByte} $. \\
Die Pixeldichte des Bildschirms beträgt bei einer Breite von $ 691,2\,\mathrm{mm} $ und einer Höhe von $ 388,8\,\mathrm{mm} $:
\begin{align}
	Pixeldichte = \frac{2560\,\mathrm{Pixel}}{27,2\,\mathrm{inch}} = 94,1\,\mathrm{ppi}
\end{align}
Mit dieser Pixeldichte wirkt das Bild scharf bis zu einer Entfernung von einem Meter.
Durch die Position ist die gezielte Brechung bei Crashs relevant für die Sicherheit von Passanten.
Daneben ist zu prüfen, ob die Radarsensorik von dem E-Papier gestört wird.\\
\subsection{LED-Streifen in der Frontschürze}
Unter den rechtlichen Kriterien gilt bei Lichtern die europäische Richtlinie ECE R48. Farbliche Lichter außerhalb der zulässigen Verbauten sind dabei nicht gestattet. Daher gilt es zu prüfen, welche weitere Möglichkeiten dieser Darstellungsart es gibt. Optionen sind beispielsweise schwarz-weiße Lichter oder farbliche Bildschirme ohne aktive Beleuchtung. \\
Für eine Farbtiefe von 256 Stufen pro Farbe benötigt jede einzelne LED 24 Bit an Speicher, da es drei mal acht Bit benötigt. Bei 332 LEDs bedeutet das einen Speicher pro Bild von $ 996\,\mathrm{Byte} $.
\begin{align}
	S &= 332\,\mathrm{Pixel} \times 24\,\frac{\mathrm{Bit}}{\mathrm{Pixel}} \cdot \\
	&=  996\,\mathrm{Byte}
\end{align}
Für ein Video von zehn Sekunden Länge wird ein Speicher S von 
\begin{align}
	S &= 996\,\mathrm{Byte} \cdot 24\,\mathrm{fps} \cdot 10\,\mathrm{s}\\
	&= 239,04\,\mathrm{kByte}
\end{align}
benötigt. Bei insgesamt 50 Inhalten bedeutet das $ 11,952\,\mathrm{MByte} $ an Speicherbedarf. \\
Eine farbige LED, die pro Pixel vier Dioden (Rot, Grün, Blau und Weiß) enthält, hat einen Leistungsaufnahme von $ 320\,\mathrm{mW} $.  
Die LED-Streifen benötigen mit einer Anzahl von 500 LED eine Leistung von $ 160\,\mathrm{W} $. Im 12 V Bordnetz beträgt die Stromstärke $ 13,3\,\mathrm{A} $. 
\subsection{E-Papiere über den vorderen Radkästen}
Bei den E-Papieren an den Fahrzeugseiten muss, wie oben erwähnt, geprüft werden, ob sie als Beleuchtung gelten.\\
Aus Sicht der Kunden ist die Lage neben den Fronttüren ideal, um Objekte anzuzeigen.
Da auf diesen E-Papieren vorwiegend Text angezeigt wird, ist ein breites und in der Höhe schmales Display von Vorteil, da dort eine Zeile leserlich angezeigt werden kann. \\
Um bei Beschädigung oder  Defekt das E-Papier auszutauschen, ist die Möglichkeit über den Zugriff vom Motorraum aus zu prüfen. \\
Das Gewicht des E-Papiers ist relativ gering und zu vernachlässigen.
Da das E-Papier nur bei Änderung des Inhaltes Strom verbraucht, muss nur eine geringe Stromversorgung sichergestellt werden. \\
Die E-Papier Embleme benötigen mit einer Auflösung von 1600 Pixel pro Zeile zu 1200 Zeilen einen Speicher pro Bild von $ 192\,\mathrm{kByte} $.
\begin{align}
	S &= \frac{1600\,\frac{\mathrm{Pixel}}{\mathrm{Zeile}} \times 1200\,\mathrm{Zeilen} \times 8\,\frac{\mathrm{Bit}}{\mathrm{Pixel}}}{10} \cdot \\
	&= 192\,\mathrm{kByte}
\end{align}
Insgesamt benötigt der Speicher eine Größe von $ 9,6\,\mathrm{MByte} $ für 50 Bilder. \\
Die Pixeldichte beträgt bei einer Breite von $ 20\,\mathrm{cm} $ ca. $ 200\,\mathrm{ppi} $. Bis zu einem halben Meter Distanz ist das Bild für den Betrachter scharf.
\subsection{LED-Streifen in den Radkästen}
Wie bei den anderen Leuchteinrichtungen im Exterieur sind hier die Marktregularien zu prüfen. Bei Möglichkeit kann auch ein eingeschränkter Modus von schwarz-weißem Licht oder nur die Anzeige im abgestellten Zustand des Fahrzeuges gewählt werden. \\
Die Verschmutzung und Beschädigung des Lichtleiters ist durch eine geeignete Konstruktion zu verhindern. \\
Aus Kundensicht muss geprüft werden, ob die Betonung des Schmutzfänger Bereichs generell gewünscht ist, oder nur in bestimmten Situationen.
Bei einer Anzahl von 200 LED pro Radkasten benötigt ein LED Streifen eine Leistung von $ 64\,\mathrm{W} $.
Ein dynamisches Lichtspiel für zehn Sekunden benötigt pro Radkasten bei einer Bildwiederholungsrate von $ 24\,\mathrm{fps} $ einen Speicher von $ 144\,\mathrm{kByte}$. \\
\begin{align}
	S &= 200\,\mathrm{Pixel} \times 24\,\frac{\mathrm{Bit}}{\mathrm{Pixel}} \cdot \\
	&= 600\,\mathrm{Byte}
\end{align}
\begin{align}
	S &= 600\,\mathrm{Byte} \cdot 24\,\mathrm{fps} \cdot 10\,\mathrm{s}\\
	&= 144\,\mathrm{kByte}
\end{align}
Für 50 Videos beträgt die Speichergröße $ 7,2\,\mathrm{MByte} $.
Die Vernetzung an dieser Stelle ist eine Herausforderung.
\subsection{Videoprojektoren in den Außenspiegeln}
Die Verfügbarkeit des Bauraums in dem Außenspiegel ist zu prüfen, indem der Bauraum von aktuellen Standbildprojektoren mit den benötigten Videoprojektoren verglichen wird. 
Daneben ist die Zulässigkeit von solchen Projektionen und deren Lichtaustrittsfläche rechtlich zu bewerten. \\
Die angestrahlte Fläche auf dem Boden reicht im besten Fall über die gesamten Seitentüren und bis zu $ 1\,\mathrm{m} $ vom Fahrzeug weg.
Der Betrachter soll bei unterschiedlichen Lichtverhältnissen noch ein möglichst kontrastreiches und helles Bild erkennen. \\
Ein Standbild benötigt bei einer Pixelanzahl pro Zeile von $ 1280 $ und $ 800 $ Zeilen ein Speichergröße von ca. $ 256\,\mathrm{kByte}$. 
\begin{align}
	S &= \frac{1280\,\frac{\mathrm{Pixel}}{\mathrm{Zeile}} \times 800\,\mathrm{Zeilen} \times 24\,\frac{\mathrm{Bit}}{\mathrm{Pixel}}}{10} \\
	&= 307,2\,\mathrm{kByte}
\end{align}
Für eine Animation von zehn Sekunden Länge benötigt eine Videoprojektor einen Speicher von $ 73,728\,\mathrm{MByte}$.
\begin{align}
	S &= 307,2\,\mathrm{kByte} \cdot 24\,\mathrm{fps} \cdot 10\,\mathrm{s}\\
	&= 73,728\,\mathrm{MByte} 
\end{align}
Insgesamt wird ein Speicher in Höhe von $ 3,6864\,\mathrm{GByte} $ benötigt. \\
Die Projektionsfläche auf dem Boden beträgt in der Breite ca. $ 1,20\,\mathrm{m} $ und in der Höhe $ 0,75\,\mathrm{m} $. Damit ergibt sich eine Pixeldichte von $ 27,1\,\mathrm{ppi} $. Die Pixeldichte reicht bis zu einer Entfernung von ca. drei Metern aus.
\subsection{Bildschirme in den hinteren Seitenfenstern}
Rechtlich ist die Zulässigkeit wegen möglicher Ablenkungen von Straßenverkehrsteilnehmern zu prüfen.
Das Verkleinern der Fensterfläche ist dahingehend zu bewerten, ob dies mit Kundeninteressen vereinbar ist.
Die Crashsicherheit an dieser Stelle und die Stabilität ist wichtig, da sich diese Komponente in unmittelbarer Nähe zu den Fahrzeuginsassen befindet.
Ein Standbild benötigt bei einer Pixelanzahl pro Zeile von $ 1280 $ und von $ 800 $ Zeilen ein Speichergröße von ca. $ 307,2\,\mathrm{kByte}$. 
\begin{align}
	S &= \frac{1280\,\frac{\mathrm{Pixel}}{\mathrm{Zeile}}\times 800\,\mathrm{Zeilen} \times 24\,\frac{\mathrm{Bit}}{\mathrm{Pixel}}}{10} \\
	&= 307,2\,\mathrm{KByte}
\end{align}
Für eine Animation von zehn Sekunden Länge benötigt das Display einen Speicher von $ 73,728\,\mathrm{MByte}$.
\begin{align}
	S &= 307,2\,\mathrm{kByte} \cdot 24\,\mathrm{fps} \cdot 10\,\mathrm{s}\\
	&= 73,728\,\mathrm{MByte}
\end{align}
Insgesamt wird ein Speicher pro Bildschirm von $ 3,6864\,\mathrm{GByte} $ benötigt. \\
Mit einer Breite von $ 30\,\mathrm{cm} $ besitzt das Display eine Pixeldichte von $ 108,4\,\mathrm{ppi} $ und bildet bis zu einer Distanz von circa einem Meter ein scharfes Bild ab.
\subsection{LED-Streifen in der Heckleuchte}
Wie bei den LED-Streifen in der Frontschürze sind die rechtlichen und wirtschaftlichen Fragestellungen gleich. \\
Grundsätzlich gilt es zu prüfen, ob die vorhandenen Steuergeräte für die Lichter Kapazitäten für dynamische Lichtstreifen frei haben.
Die weiteren technischen Kriterien verhalten sich wie oben erläutert. \\
Für ein Bild benötigt man einen Speicher von $ 1,173\,\mathrm{kByte} $.
\begin{align}
	S &= 391\,\mathrm{Pixel} \times 24\,\frac{\mathrm{Bit}}{\mathrm{Pixel}} \cdot \\
	&= 1,173\,\mathrm{kByte}
\end{align}
Für eine Animation von zehn Sekunden Länge sind $ 281,52\,\mathrm{kByte} $ erforderlich.
\begin{align}
	S &= 1,173\,\mathrm{kByte} \cdot 24\,\mathrm{fps} \cdot 10\,\mathrm{s} \\
	&= 281,52\,\mathrm{kByte}
\end{align}
Insgesamt wird ein Speicher von $ 14,076\,\mathrm{MByte} $ benötigt.
Der Strombedarf für das gesamte Lichtband beträgt ca. $ 128\,\mathrm{W} $
\subsection{E-Papier in der Heckleuchte}
Die E-Papiere in der Heckleuchte haben ähnliche Eigenschaften wie die E-Papiere über den vorderen Seitenkästen in Bezug auf die rechtlichen Kriterien. Durch die Position in der Nähe der Heckleuchten kann von dort die Stromversorgung und Busanbindung erfolgen. Der Strombedarf und Datenspeicherbedarf von $ 160\,\mathrm{kByte} $ pro Bild sind gleich mit dem anderen E-Papier. 
\begin{align}
	S &= \frac{1600\,\mathrm{Pixel pro Zeile} \times 1200\,\mathrm{Zeilen} \times 8\,\frac{\mathrm{Bit}}{\mathrm{Pixel}}}{10} \cdot \\
	&= 192\,\mathrm{kByte}
\end{align}
Für 50 Bilder sind es dementsprechend $ 9,6\,\mathrm{MByte} $.
Die Pixeldichte beträgt das Gleiche wie beim Emblem am Radkasten.
\section{Analyse der Interieur-Komponenten}
Im folgenden werden die einzelnen Interieur-Komponenten mit den Kriterien, die in \ref{cha:Kriterien} definiert wurden, analysiert.
\subsection{LED-Streifen im Interieur}
Rechtlich ist bei den Animationen durch LED-Streifen im Fahrzeug die Vermeidung von Ablenkung des Fahrers oder der Fahrerin zu gewährleisten.
Durch die Länge des Streifens müssen für die Crashsicherheit eventuelle Sollbruchstellen sichergestellt werden. \\
Bei einer Anzahl von 719 LED benötigt der LED Streifen eine Leistung von $ 230\,\mathrm{W} $.
Für eine Inszenierung von zehn Sekunden Länge benötigt der LED-Streifen Speicherplatz in der Höhe von $ 25,884\,\mathrm{MByte}$.
\begin{align}
	S &= 719\,\mathrm{Pixel} \times 24\,\frac{\mathrm{Bit}}{\mathrm{Pixel}} \cdot \\
	&= 2,157\,\mathrm{kByte}
\end{align}
\begin{align}
	S &= 2,157\,\mathrm{kByte} \cdot 24\,\mathrm{fps} \cdot 10\,\mathrm{s}\\
	&= 517,68\,\mathrm{kByte}
\end{align}
Für 50 Videos beträgt die Speichergröße $ 25,884\,\mathrm{MByte} $.
\subsection{LED-Türtafeln}
Rechtlich gesehen ist zu prüfen, ab welchem Helligkeitsgrad und ab welcher Farbinszenierung der Fahrer oder die Fahrerin durch die LED zu stark abgelenkt werden. \\
Im Moment wird nur eine LED pro Türtafel genutzt, wodurch Möglichkeiten der Inszenierung im Vergleich zu einer LED-Matrix nicht genutzt werden. 
Die LEDs benötigen eine Leistung von $ 320\,\mathrm{mW} $ pro Türe.
Die Ansteuerung der Türtafeln kann durch die Nähe zu den LED-Streifen im Interieur zusammen gesteuert werden.
\subsection{Bildschirme in der Einstiegsleiste}
Unter den rechtlichen Aspekten gibt es in einer ersten Untersuchung keine Beschränkungen, da der Bildschirm nur bei geöffneter Türe und dementsprechend im Stand betrachtet werden kann.
Die technische Umsetzbarkeit ist durch Bestätigung in ersten Voruntersuchungen, wie zum Beispiel Kugelfalltest, positiv. Technische Herausforderungen an dieser Stelle sind Feuchtigkeitsschutz, Temperaturempfindlichkeit, und Kratzfestigkeit. \\
Ein Standbild benötigt bei einer Pixelanzahl pro Zeile von $ 1280 $ und von $ 1024 $ Zeilen ein Speichergröße von ca. $ 393,216\,\mathrm{kByte}$. 
\begin{align}
	S &= \frac{1280\,\frac{\mathrm{Pixel}}{\mathrm{Zeile}}\times 1024\,\mathrm{Zeilen} \times 24\,\frac{\mathrm{Bit}}{\mathrm{Pixel}}}{10} \\
	&= 393,216\,\mathrm{KByte}
\end{align}
Für eine Animation von zehn Sekunden Länge benötigt das Display einen Speicher von $ 94,37184\,\mathrm{MByte}$.
Insgesamt bei 50 Videos beträgt der benötigte Speicher $ 4,718592\,\mathrm{GByte}$. \\
Bei einer Breite von $ 30\,\mathrm{cm} $ beträgt die Pixeldichte $ 108,4\,\mathrm{ppi} $. Damit ist die Betrachtung von bis zu einem Meter scharf.
\subsection{Videoprojektoren im Fußraum}
Unter den rechtlichen Fragestellungen ergibt sich die Vermeidung von Ablenkung für den Fahrer oder die Fahrerin.
Aus Kundenperspektive ist die angestrahlte Oberfläche eventuell nicht geeignet, da diese stärker verschmutzt sein kann. \\
Unter den technischen Kriterien ist die Verfügbarkeit des Bauraums mit der benötigten Kühlung zu prüfen. Eine Herausforderung ist die Befestigung des Videoprojektors, da dieser sich bei Fahrt relativ zum Auto nicht bewegen darf. 
Ein Standbild benötigt bei einer Pixelanzahl pro Zeile von $ 1280 $ und von $ 800 $ Zeilen ein Speichergröße von ca. $ 307,2\,\mathrm{kByte}$. 
\begin{align}
	S &= \frac{1280\,\frac{\mathrm{Pixel}}{\mathrm{Zeile}}\times 800\,\mathrm{Zeilen} \times 24\,\frac{\mathrm{Bit}}{\mathrm{Pixel}}}{10} \\
	&= 307,2\,\mathrm{kByte}
\end{align}
Für eine Animation von zehn Sekunden Länge benötigt das Display einen Speicher von $ 73,728\,\mathrm{MByte}$.
Insgesamt bei 50 Videos beträgt der benötigte Speicher $ 3,6864\,\mathrm{GByte}$.
Die Pixeldichte beträgt bei einer Breite von $ 30\,\mathrm{cm} $ $ 108,4\,\mathrm{ppi} $. Damit ist die Betrachtung von bis zu einem Meter scharf.
\subsection{Morphende Oberfläche in der Mittelkonsole}
Die tatsächlich verwendete technische Realisierung der morphenden Oberfläche ist eine Übergangslösung für eine neue Technik, die einzelne Pixelelemente auf einer Fläche gesteuert hoch und herunterfahren kann. Diese Technik ist derzeit in der Entwicklung. Wie die Farbtiefe bei einem Bildschirm kann die Höhe der einzelnen Punkte eingestellt werden. \\
Mit 8 Bit Höhentiefe benötigt ein Feld mit 20 auf 20 einzelnen Punkten einen Speicher in der Größe von $ 400\,\mathrm{Byte}$.
Eine weitere technische Weiterentwicklung kann Animationen auf der Mittelkonsole abspielen. Die Datenmenge steigert sich dementsprechend mit der Bildwiederholungsrate und der Dauer der Animation. Bei einer Bildwiederholungsrate von $ 24\,\mathrm{fps} $ und 10 Sekunden Dauer beträgt der benötigte Speicher $ 96\,\mathrm{kByte}$.
\subsection{Durchsichtiger Bildschirm im Dachfenster} 
Herausforderungen bestehen durch die Temperaturentwicklung der Oberfläche durch Sonneneinstrahlung und die Festigkeit der Kontaktierung bei Bewegung des Glasdaches. 
Ein Standbild benötigt bei einer Pixelanzahl pro Zeile von $ 1920 $ und von $ 1080 $ Zeilen ein Speichergröße von ca. $ 207,36\,\mathrm{kByte}$. 
\begin{align}
	S &= \frac{1920\,\frac{\mathrm{Pixel}}{\mathrm{Zeile}}\times 1080\,\mathrm{Zeilen} \times 8\,\frac{\mathrm{Bit}}{\mathrm{Pixel}}}{10} \\
	&= 207,36\,\mathrm{kByte}
\end{align}
Für eine Animation von zehn Sekunden Länge benötigt das Display einen Speicher von $ 49,7664\,\mathrm{MByte}$.
Insgesamt bei 50 Videos beträgt der benötigte Speicher $ 2,488320\,\mathrm{GByte}$.
Mit einer Breite von $ 80\,\mathrm{cm} $ und $ 1920 $ Pixel pro Zeile beträgt die Pixeldichte $ 61,0\,\mathrm{ppi} $. Das Bild ist bis zu ca. $ 1,5\,\mathrm{m} $ scharf.
\subsection{LED-Matrix im Dachhimmel}
Ein Standbild benötigt bei einer Pixelanzahl pro Zeile von $ 192 $ und von $ 96 $ Zeilen ein Speichergröße von ca. $ 55,296\,\mathrm{kByte}$. 
\begin{align}
	S &= 192\,\frac{\mathrm{Pixel}}{\mathrm{Zeile}}\times 96\,\mathrm{Zeilen} \times 24\,\frac{\mathrm{Bit}}{\mathrm{Pixel}} \\
	&= 55,296\,\mathrm{kByte}
\end{align}
Für eine Animation von zehn Sekunden Länge benötigt das Display einen Speicher von $ 13,27104\,\mathrm{MByte}$.
Insgesamt bei 50 Videos beträgt der benötigte Speicher $ 663,552\,\mathrm{MByte}$.
\subsection{Duftflakons im Innenraum}
Eine technische Verbreitung unterschiedlicher individueller Düfte ist in diesem Konzept nicht realisiert. Ein aktueller technischer Aufbau besteht aus einem Set mit unterschiedlichen Duftflakons, die einzeln geöffnet werden können. Dadurch ist die Anzahl der Düfte beschränkt auf die Set-Größe.
\subsection{Bildschirmoberflächen im Cockpit}
Das Anzeigen individualisierter Bildschirmoberflächen auf vorhandenen Displays ist unter den oben beschrieben Kriterien schon im Fahrzeug möglich und benötigt keine weitere Analyse.
\subsection{Soundplayer im Innenraum}
Das Abspielen individualisierter Sounds durch vorhandene Lautsprecher ist unter den oben beschrieben Kriterien schon im Fahrzeug möglich und benötigt keine weitere Analyse.
\section{Zusammenfassung Analyse} \label{ZusammendassungAnalyse}
In der Tabelle \ref{tab:Speicherbedarf} werden die Komponenten, die visuelle oder haptische Anzeigen sind, mit dem benötigten Speicher für die Anzeige eines Bildes dargestellt. Die Berechnung erfolgt mit der Gleichung \ref{eq:Bilddatenmenge}.
\begin{table}[hbt]	
	\centering
	\renewcommand{\arraystretch}{1.5}	% Skaliert die Zeilenhöhe der Tabelle
	\captionabove[Berechnung des Speicherbedarfes für ein Bild]{Berechnung des Speicherbedarfes für ein Bild}
	\label{tab:Speicherbedarf}
	\begin{tabular}{c|cccc}
		\textbf{Komponente} & \parbox[t]{0.1\linewidth}{\centering Pixel} & \parbox[t]{0.11\linewidth}{\centering Bit pro Pixel} & \parbox[t]{0.15\linewidth}{\centering Kompression} & \parbox[t]{0.1\linewidth}{\centering Speicher pro Bild} \\ 
		\hline 
		\hline 
		\parbox[t]{0.3\linewidth}{\centering E-Papier in der Frontschürze} & $ 2560 \times 1440 $ & $ 8 $ & $ 10 $ & $ 368,64\,\mathrm{kByte} $\\ \parbox[t]{0.3\linewidth}{\centering E-Papier über den vorderen Radkästen} & $ 1600 \times 1200 $ & $ 8 $ & $ 10 $ & $ 192\,\mathrm{kByte} $ \\
		\parbox[t]{0.3\linewidth}{\centering E-Papier in der Heckleuchte} & $ 1600 \times 1200 $ & $ 8 $ & $ 10 $ & $ 192\,\mathrm{kByte} $ \\
		\parbox[t]{0.3\linewidth}{\centering LED-Streifen in der Frontschürze} & $ 332 \times 1 $ & $ 24 $ & $ 1 $ & $ 996\,\mathrm{Byte} $ \\
		\parbox[t]{0.3\linewidth}{\centering LED-Streifen in den Radkästen} & $ 200 \times 1 $ & $ 24 $ & $ 1 $ & $ 600\,\mathrm{Byte} $\\ \parbox[t]{0.3\linewidth}{\centering LED-Streifen in der Heckleuchte} & $ 391 \times 1 $ & $ 24 $ & $ 1 $ & $ 1,173\,\mathrm{kByte} $ \\ 
		\parbox[t]{0.3\linewidth}{\centering LED-Streifen im Interieur} & $ 719 \times 1 $ & $ 24 $ & $ 1 $ & $ 2,157\,\mathrm{kByte} $ \\
		\parbox[t]{0.3\linewidth}{\centering LED Türtafeln} & $ 4 \times 1 $ & $ 24 $ & $ 1 $ &  $ 12\,\mathrm{Byte} $ \\
		\parbox[t]{0.3\linewidth}{\centering Videoprojektoren in\\den Außenspiegeln} & $ 1280 \times 800 $ & $ 24 $ & $ 10 $ &   $ 307,2\,\mathrm{kByte} $ \\ 
		\parbox[t]{0.3\linewidth}{\centering Videoprojektoren im Fußraum} & $ 1280 \times 800 $ & $ 24 $ & $ 10 $ & $ 307,2\,\mathrm{kByte} $ \\
		\parbox[t]{0.3\linewidth}{\centering Bildschirme in den\\hinteren Seitenfenstern} & $ 1280 \times 800 $ & $ 24 $ & $ 10 $ & $ 307,2\,\mathrm{kByte} $ \\
		\parbox[t]{0.3\linewidth}{\centering Bildschirme in der Einstiegsleiste} &  $ 1280 \times 1024 $ & $ 24 $ & $ 10 $ & $ 393,216\,\mathrm{kByte} $ \\
		\parbox[t]{0.3\linewidth}{\centering Durchsichtiger Bildschirm\\im Dachfenster} & $ 1920 \times 1080 $ & $ 8 $ & $ 10 $ & $ 207,36\,\mathrm{kByte} $ \\
		\parbox[t]{0.3\linewidth}{\centering LED-Matrix im Dachhimmel} & $ 192 \times 96 $ & $ 24 $ & $ 1 $ & $ 55,296\,\mathrm{kByte} $ \\
		\parbox[t]{0.3\linewidth}{\centering Morphende Oberfläche\\in der Mittelkonsole} & $ 20 \times 20 $ & $ 8 $ & $ 1 $ & $ 400\,\mathrm{Byte} $ \\
	\end{tabular} 
\end{table}
Die unterschiedlichen Datenmengen zwischen $ 12\,\mathrm{Byte} $ und $ 393,216\,\mathrm{kByte} $ für ein Bild machen deutlich, dass eine Abwägung zwischen hoher Pixelanzahl und Farbtiefe gemacht werden muss. Die benötigte Speichergröße wird neben dem Speicher pro Bild in der Tabelle \ref{tab:Werte} dargestellt. Es wird deutlich, dass das Anzeigen von bewegten Bildern mehr Daten benötigt.
\begin{table}[hbt]	
	\centering
	\renewcommand{\arraystretch}{1.5}	% Skaliert die Zeilenhöhe der Tabelle
	\captionabove[Liste der Komponenten mit den oben festgestellten Werten]{Liste der Komponenten mit den oben festgestellten Werte}
	\label{tab:Werte}
	\begin{tabular}{c|ccc}
		\textbf{Komponente} & \parbox[t]{0.18\linewidth}{\centering Speicher pro Bild} & \parbox[t]{0.18\linewidth}{\centering Speicher pro Video} &\parbox[t]{0.18\linewidth}{\centering Speichergröße gesamt} \\ 
		\hline 
		\hline 
		\parbox[t]{0.3\linewidth}{\centering E-Papier in der Frontschürze} & $ 368,64\,\mathrm{kByte} $ && $ 18,432\,\mathrm{MByte} $  \\
		\parbox[t]{0.3\linewidth}{\centering E-Papier über den vorderen Radkästen} & $ 192\,\mathrm{kByte} $ && $ 9,6\,\mathrm{MByte} $ \\
		\parbox[t]{0.3\linewidth}{\centering E-Papier in der Heckleuchte} & $ 192\,\mathrm{kByte} $ && $ 9,6\,\mathrm{MByte} $ \\
		\parbox[t]{0.3\linewidth}{\centering LED-Streifen in der Frontschürze} & $ 996\,\mathrm{Byte} $ & $ 239,04\,\mathrm{kByte} $ & $ 11,952\,\mathrm{MByte} $ \\
		\parbox[t]{0.3\linewidth}{\centering LED-Streifen in den Radkästen} & $ 600\,\mathrm{Byte} $ & $ 144\,\mathrm{kByte} $ & $ 7,2\,\mathrm{MByte} $ \\ 
		\parbox[t]{0.3\linewidth}{\centering LED-Streifen in der Heckleuchte} & $ 1,173\,\mathrm{kByte} $ & $ 281,52\,\mathrm{kByte} $ & $ 14,076\,\mathrm{MByte} $ \\ 
		\parbox[t]{0.3\linewidth}{\centering LED-Streifen im Interieur} & $ 2,157\,\mathrm{kByte} $ & $ 517,68\,\mathrm{kByte} $ &$ 25,884\,\mathrm{MByte} $ \\
		\parbox[t]{0.3\linewidth}{\centering LED Türtafeln} & $ 12\,\mathrm{Byte} $ & $ 2,88\,\mathrm{kByte} $ & $ 144\,\mathrm{KByte} $ \\
		\parbox[t]{0.3\linewidth}{\centering Videoprojektoren in\\den Außenspiegeln} & $ 307,2\,\mathrm{kByte} $ & $ 73,728\,\mathrm{MByte} $ & $ 3,6864\,\mathrm{GByte} $ \\ 
		\parbox[t]{0.3\linewidth}{\centering Videoprojektoren im Fußraum} & $ 307,2\,\mathrm{kByte} $ & $ 73,728\,\mathrm{MByte} $ &$ 3,6864\,\mathrm{GByte} $ \\
		\parbox[t]{0.3\linewidth}{\centering Bildschirme in den\\hinteren Seitenfenstern} & $ 307,2\,\mathrm{kByte} $ & $ 73,728\,\mathrm{MByte} $ & $ 3,6864\,\mathrm{GByte} $ \\
		\parbox[t]{0.3\linewidth}{\centering Bildschirme in der Einstiegsleiste} & $ 393,216\,\mathrm{kByte} $ & $ 94,37184\,\mathrm{MByte} $ & $ 4,718592\,\mathrm{GByte} $ \\
		\parbox[t]{0.3\linewidth}{\centering Durchsichtiger Bildschirm\\im Dachfenster} & $ 207,36\,\mathrm{kByte} $ & $ 49,7664\,\mathrm{MByte} $ & $ 2,48832\,\mathrm{GByte} $ \\
		\parbox[t]{0.3\linewidth}{\centering LED-Matrix im Dachhimmel} & $ 55,296\,\mathrm{kByte} $ & $ 13,27104\,\mathrm{MByte} $ & $ 663,552\,\mathrm{MByte} $ \\
		\parbox[t]{0.3\linewidth}{\centering Morphende Oberfläche\\in der Mittelkonsole} & $ 400\,\mathrm{Byte} $ & $ 96\,\mathrm{kByte} $ &$ 4,8\,\mathrm{MByte} $ \\
	\end{tabular} 
\end{table}
In diesem Abschnitt werden die Komponenten als gesamtes unter den wirtschaftlichen Kriterien betrachtet, weil die Komponenten nicht als einzelne Sonderausstattung, sondern als ganzes oder gebündelt vertrieben werden sollen. Daher ist eine Gesamtbetrachtung der wirtschaftlichen Kriterien sinnvoll. \\
In Tabelle \ref{tab:Entwicklung} werden alle Komponenten mit den nötigen Anpassungen in der Serienentwicklung in den bestimmten Bereichen auf Basis der vorherigen technischen Analyse dargestellt. \\
Für eine erste Einschätzung der Entwicklungskosten der Komponenten werden die nötigen Anpassungen in der Fahrzeugentwicklung dargestellt. Die Anpassungen an bestehende Fahrzeuge sind von großer Bedeutung, da diese hohe Kosten und lange Zeiten in Anspruch nehmen. Die reinen Entwicklungskosten für die technischen Komponenten sind in diesem Zustand noch nicht abzusehen.
\begin{table}[hbt]	
	\centering
	\renewcommand{\arraystretch}{1.5}	% Skaliert die Zeilenhöhe der Tabelle
	\captionabove[Liste der Komponenten mit den nötigen Eingriffen in die Fahrzeugentwicklung]{Liste der Komponenten mit den nötigen Eingriffen in die Fahrzeugentwicklung}
	\label{tab:Entwicklung}
	\begin{tabular}{c|ccc}
		\textbf{Komponente} & \textbf{Hardware} & \textbf{E/E Architektur} & \textbf{Software} \\ 
		\hline 
		\hline 
		\parbox[t]{0.4\linewidth}{\centering E-Papier in der Frontschürze} & X & X & X \\
		\parbox[t]{0.4\linewidth}{\centering LED-Streifen in der Frontschürze} & X & X & X \\
		\parbox[t]{0.4\linewidth}{\centering E-Papier über den vorderen Radkästen} & X & X & X \\
		\parbox[t]{0.4\linewidth}{\centering LED-Streifen in den Radkästen} & X & X & X \\
		\parbox[t]{0.4\linewidth}{\centering Videoprojektoren in\\den Außenspiegeln} & X & X & X \\
		\parbox[t]{0.4\linewidth}{\centering Bildschirme in den\\hinteren Seitenfenstern} & X & X & X \\
		\parbox[t]{0.4\linewidth}{\centering LED-Streifen in der Heckleuchte} &  & X & X \\
		\parbox[t]{0.4\linewidth}{\centering E-Papier in der Heckleuchte} & X & X & X \\
		\parbox[t]{0.4\linewidth}{\centering LED-Streifen im Interieur} &  & X & X \\
		\parbox[t]{0.4\linewidth}{\centering LED Türtafeln} &  & X & X \\
		\parbox[t]{0.4\linewidth}{\centering Bildschirme in der Einstiegsleiste} &  & X & X \\
		\parbox[t]{0.4\linewidth}{\centering Videoprojektoren im Fußraum} &  & X & X \\
		\parbox[t]{0.4\linewidth}{\centering Morphende Oberfläche\\in der Mittelkonsole} &  & X & X \\
		\parbox[t]{0.4\linewidth}{\centering Durchsichtiger Bildschirm\\im Dachfenster} &  & X & X \\
		LED-Matrix im Dachhimmel & X & X & X \\
		\parbox[t]{0.4\linewidth}{\centering Bildschirmoberflächen im Cockpit} &  &  & X \\
		\parbox[t]{0.4\linewidth}{\centering Soundplayer im Innenraum} &  & X & X \\
		\parbox[t]{0.4\linewidth}{\centering Duftflakons im Innenraum} &  & X & X \\
	\end{tabular} 
\end{table}
		%Der parbox-Befehl ist erforderlich, damit ein Zeilenumbruch erzeugt werden kann. c-Spalten (zentriert) erlauben nicht automatisch einen Zeilenumpruch. Linksbündig gesetzte p-Spalten erlauben automatisch den Zeilenumbruch.%

