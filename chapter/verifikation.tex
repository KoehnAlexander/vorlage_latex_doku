\chapter{Verifikation und Diskussion}
\label{cha:Verifikation}
Abgesehen von den Konzeptentwürfen für die informationstechnische Anbindung der Komponenten, versucht das Kapitel \ref{cha:Analyse} erste Erkenntnisse zu erzielen, wie die Komponenten in eine Serie implementiert werden können. Diese Analyse ist in der Durchführung herausfordernd, da durch den frühen Stadium des Forschungsprojekts grundlegende Fragestellungen in der technischen Realisierung ungelöst sind. Dadurch kann diese Analyse in manchen Bereichen nur erste Informationen zu den Komponenten sammeln. Die Speicherdimensionierung anhand von Berechnungen einzelner Bilder und Videos ist daher nur eine erste Einschätzung, da auch dort Weiterentwicklungen Einflüsse haben.\\
Die Konzeptentwürfe sind erste Ansätze für eine mögliche Realisierung mit der aktuellen Technik. Da die Projektidee und die Ausführung erst weit in der Zukunft veranschlagt sind, können zu diesem späteren Zeitpunkt neue technische Möglichkeiten für die Implementierung der Komponenten vorhanden sein. Besonders die Nutzung von drahtlosen Verbindungen zur Kommunikation von Steuergeräten in Automobilen wurden noch nicht betrachtet. 
%TODO%

... Verifikation, Auswertung, Lösungsbewertung, Diskussion der Ergebnisse
