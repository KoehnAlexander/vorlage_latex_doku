\chapter{Zusammenfassung}
\label{cha:zusammenfassung}
Mit den gesammelten Erkenntnisse aus der Analyse der Komponente kommt diese Arbeit zu der Erkenntnis, dass eine technische Realisierung der Komponenten möglich ist. \\
Mit den beschriebenen Grundlagen in \ref{cha:Grundlagen} kann eine wissenschaftliche Bewertung der Komponenten nach objektiven Maßstab getroffen werden. Die Grundlagen und die Analyse vermitteln ein Verständnis für die Komplexität eines Forschungsprojektes, in dem nicht nur technische Aspekte eine Rolle spielen, sondern auch rechtliche, wirtschaftliche und optische Themen. \\
Die Zusammenhänge zwischen technischer Entwicklung und Sinnlicher Erfahrung wurden zum Beispiel mit dem Auflösungsvermögen des Auges näher betrachtet, worauf sich die Speicherberechnungen wiederum beziehen. \\
Besonders in der informationstechnischen Anbindung kann diese Arbeit für die Weiterarbeit in dem Projekt eine Grundlage für Serienimplementierungen bieten, da in \ref{cha:Konzeptentwurf} Konzeptentwürfe erstellt wurden. \\
Die Anbindung der Komponenten an bestehende Bordnetzsysteme ist abhängig vom gewünschten Szenario im Kundenverhalten. Je nach Ausführung benötigen Komponenten unterschiedlich starke Mikrocontroller und Busanbindungen.
%TODO Zusammenfassung fertig machen%
