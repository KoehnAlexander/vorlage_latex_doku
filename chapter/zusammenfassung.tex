\chapter{Zusammenfassung}
\label{cha:zusammenfassung}
Mit den gesammelten Erkenntnisse aus der Analyse der Komponenten kommt diese Arbeit zu der Erkenntnis, dass eine technische Realisierung möglich ist. \\
Mit den beschriebenen Grundlagen in Kapitel \ref{cha:Grundlagen} kann eine wissenschaftliche Bewertung der Komponenten nach objektivem Maßstab getroffen werden. Die Grundlagen und die Analyse vermitteln ein Verständnis für die Komplexität eines Forschungsprojektes, in dem nicht nur technische Aspekte eine Rolle spielen, sondern auch rechtliche, wirtschaftliche und optische Themen. \\
Die Zusammenhänge zwischen technischer Entwicklung und sinnlicher Erfahrung wurden zum Beispiel mit dem Auflösungsvermögen des Auges näher betrachtet, worauf sich die Speicherberechnungen wiederum beziehen. \\
Besonders in der informationstechnischen Anbindung kann diese Arbeit für die Weiterarbeit in dem Projekt eine Grundlage für Serienimplementierungen bieten, da in Kapitel \ref{cha:Konzeptentwurf} Konzeptentwürfe erstellt wurden. \\
Die Anbindung der Komponenten an bestehende Bordnetzsysteme ist abhängig vom gewünschten Szenario im Kundenverhalten. Je nach Ausführung benötigen Komponenten unterschiedliche Mikrocontroller und Busanbindungen.
Für eine technische Realisierung benötigt das System ein leistungsstarkes Bussystem, wie zum Beispiel MOST oder Automotive Ethernet, damit die Inhalte in einer möglichst geringen Zeit angezeigt werden können. 
