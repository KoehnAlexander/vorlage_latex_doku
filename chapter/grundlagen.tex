\chapter{Grundlagen}
\label{cha:Grundlagen}
Im folgenden werden für diese Arbeit notwendige Grundlagen zu unterschiedlichen Technologien und Wissensbereiche erarbeitet. 
Zuerst wird die menschliche Wahrnehmung erläutert, da diese die Grundlage für eine Evaluierung von Komponenten bietet, wenn diese Komponenten die Sinne des Nutzers ansprechen sollen. Anschließend werden die Technologien der Komponenten des Prototypen vorgestellt, um einen Einblick auf die Funktionsweise und wichtigen Kriterien zur Bewertung der Technologien und Techniken zu liefern.
\section{Menschliche Wahrnehmung}
Menschliche Wahrnehmung ist die \glqq Tätigkeit oder Vorgang der Informationsaufnahme durch unsere Sinne\grqq{} \cite[Seite 12]{Buhler.2017}. Dieser Prozess beschränkt sich dabei nicht nur auf die Aufnahme von Informationen, sondern auch auf die Auswahl und Bewertung der Informationsdaten nach Relevanz. Die Aufnahme der Informationen geschieht über Sinnesorgane, die die Informationen über unterschiedliche Techniken in Reize wandeln. Die Auswahl und Bewertung erfolgt hauptsächlich im zentralen Nervensystem und dem Gehirn. \cite[Vgl. Seite 12]{Buhler.2017}\\
Mit 70 \% der wahrgenommenen Umweltreize ist das Auge, das bedeutendste Sinnesorgan des Menschen vor der Haut, Nase, Ohr oder Zunge. Unsere Wahrnehmung ist dabei immer eine Interpretation der erhaltenen Sinnesreize aller Sinnesorgane. Unsere visuelle Wahrnehmung ist daher nicht nur durch die Augen bestimmt, sondern auch durch die Ohren, der Nase, der Zunge und der Haut. Daneben spielen unsere Erfahrungen und emotionale Lage einen Einfluss auf die Wahrnehmung. \cite[Vgl. Seite 13 f.]{Buhler.2017}
Das bedeutet, dass die Wahrnehmung ganzheitlich betrachtet werden muss, da die einzelnen Wahrnehmungsarten miteinander in Wechselwirkung stehen.\\
Mit Medien können visuelle, auditive, haptische, motorische und olfaktorische Sinneskanäle angesprochen werden. Das Ziel der Medien ist dabei die Aufmerksamkeit des Menschen auf das Objekt zu richten und den erwünschten Einfluss auf den Menschen zu schaffen. Visuelle Inhalte können Schriften, Grafiken, Animationen oder Farben sein. Auditive sind Musik oder Geräusche. Haptische Inhalte sind fühlbare Strukturen und Oberflächen, während motorische Inhalte bewegliche Teile sind. Olfaktorische Reize sind Düfte und Gerüche. \cite[Vgl. Seite 3]{Buhler.2017}\\
In den nächsten Unterkapiteln werden die unterschiedlichen menschlichen Wahrnehmungsarten vertieft erläutert, wobei der Fokus auf die visuelle Wahrnehmung gerichtet ist, da dort der Schwerpunkt der späteren Arbeit liegt. Zu allen Wahrnehmungen wird auf der Physiologie der Sinne eingegangen und für diese Arbeit relevante Details.
\subsection{Visuelle Wahrnehmung}
Die visuelle Wahrnehmung basiert hauptsächlich auf den Sinneseindrücken durch unser Auge, das lichtempfindliche Zellen besitzt und daneben wie oben erwähnt, auch aus dem Zusammenspiel der anderen Sinnesorgane.\\
Die Zellen werden dabei zwischen Stäbchen und Zapfen unterschieden. Die Mehrzahl an Zellen bilden die spektral unempfindlichen Stäbchen, mit ca. 120 Millionen pro Auge, während nur eine geringe Anzahl von 7 Millionen pro Auge farbempfindliche Zapfen sind. Durch den Unterschied in der Anzahl ist das Sehen bei Dunkelheit eher schwarz-weiß, da die Anzahl der Zapfen nicht ausreicht, um genügend Licht zu erhalten und ein Farbbild zu erzeugen. Dabei ist ein Zapfen immer nur für eine der drei licht Frequenzbereiche rot (langwellig), grün (mittelwellig) oder blau (kurzwellig) lichtempfindlich. Die Farben in der menschlichen Wahrnehmung sind daher ein Ergebnis der Signalverarbeitung der drei unterschiedlichen Zapfenarten. \cite[Vgl. Seite 14]{Buhler.2017}\\
Menschen können dabei nicht von ihrem Standpunkt aus den gesamten Raum betrachten, sondern durch biologischen Gegebenheiten immer nur ein Feld, das in der Horizontalen ca. 180\,° abdeckt und in der Vertikalen 120\,°. Von diesem Blickfeld sind nur ca. 1,5\,° in beiden Dimensionen als scharfes Bild abgedeckt. Durch Bewegungen des Auges und des Kopfes werden verschiedene scharfe Bereiche abgedeckt, die unser Gehirn zusammenfügt, um ein ganzheitliches scharfes Blickbild zu erzeugen. \cite[Vgl. Seite 14]{Buhler.2017}
%\cite[Vgl. Seite 142 ff.]{Schonhammer.2013}%
\subsection{Haptische Wahrnehmung}
Die haptische Wahrnehmung ist ein Teilbereich der Somatosensorik. Die Sinneszellen der Somatosensorik werden in drei Bereiche eingeteilt:
\begin{itemize}
	\item Exterozeption, Wahrnehmung der Außenwelt
	\item Propriozeption, Wahrnehmung der Stellung der Gliedmaßen
	\item Interozeption, Wahrnehmung des inneren Körpers
\end{itemize}
Die somatosensorische Wahrnehmung verbindet diese drei Arten, die zum Teil bewusst oder unbewusst vom Körper aufgenommen werden. Besonders relevant für diese Arbeit im Bereich der haptischen Wahrnehmung ist die Exterozeption beziehungsweise die Oberflächensensibilität. \cite[Vgl. Seite 26]{Sprenger.2020}\\
Die haptische Wahrnehmung erfolgt durch Rezeptoren in der Haut, die die Form, Oberfläche und Position von Objekten registrieren. Die Rezeptoren können unterschieden werden in
\begin{itemize}
	\item Thermorezeptoren für relative Temperaturunterschiede zur Körperbefindlichkeit, 
	\item Chemorezeptoren für Stoffe,
	\item Nozirezeptoren für starke Temperaturunterschiede oder Drücke bis zur Gewebeschädigung und
	\item Mechanorezeptoren für Empfindung von Oberflächen und Druck.
\end{itemize}
Die Mechanorezeptoren können wiederum in unterschiedliche Arten eingeteilt werden, die auf Druck, Berührung oder Vibration reagieren. \cite[Vgl. Seite 26 f.]{Sprenger.2020}\\
Die Verteilung der unterschiedlichen Rezeptoren ist im Körper ungleichmäßig. In der Handinnenflächen gibt es zum Beispiel Areale mit unterschiedlicher Empfindsamkeit \glqq auf Druckintensität, Geschwindigkeit einer Veränderung an der Haut oder einer Vibration. \grqq{} \cite[Seite 29]{Sprenger.2020}\\
Die Empfindungsschwelle gibt Auskunft darüber, wie stark eine Hautstelle gedrückt werden muss, damit eine Berührung wahrgenommen wird. Durch das Berühren der Haut mit zwei Tastpunkten in bestimmter Entfernung kann das räumliche Auflösungsvermögen bestimmt werden. \cite[Vgl. Seite 28]{Sprenger.2020}\\
Die Wahrnehmung von Oberflächen geschieht über alle Sinneseindrücke. Zuerst werden die Hauptmerkmale Oberflächenstruktur, wie glatt oder rau, wahrgenommen und die Größe des Objektes durch die visuelle Wahrnehmung erfasst und im zweiten Schritt die gefühlte Temperatur. \cite[Vgl. Seite 33]{Sprenger.2020}\\
\glqq Die Verknüpfung der haptischen Wahrnehmung mit projizierten Simulationen zeigt letztendlich den Wunsch nach Kombinationen zu multisensuellen Interfaces, die allerdings nicht mehr die audiovisuelle Wahrnehmung als strikt dominant betrachten und die haptische Wahrnehmung ausschließlich als unterstützenden Sinn zur verbesserten Immersion heranziehen. Über die Haptik sollen direkt und unabhängig von anderen Sinnen Informationen gegeben wie auch erfahren werden, die in Kombination mit den anderen Sinnen kräftigere Informationsträger sind und somit auch als multisensuelle Kombinationen erforscht werden. \grqq{} \cite[Seite 263]{Sprenger.2020}
\subsection{Akustische Wahrnehmung}
Der Mensch ist in der Lage mit den Ohren Schallwellen zwischen 20 Hertz und 20 Kilohertz zu hören. Schallwellen sind Druckwellen eines Mediums wie Luft oder Wasser in longitudinaler Richtung. Das bedeutet, die Druckamplitudenrichtung ist parallel zur Ausbreitungsrichtung. \cite[Vgl. Seite 217]{Schonhammer.2013} \\
\glqq Über den äußeren Gehörgang gelangt die Schallwelle zum Trommelfell. Die Schwingungen des Trommelfells werden über die Kette der drei Gehörknöchelchen (Hammer, Amboss, Steigbügel) an das ovale Fenster übertragen, dessen Membran die mit Flüssigkeit gefüllte Schnecke (Cochlea) abschließt. Die Cochlea ist ein schneckenförmiger Kanal, der in das überaus harte Felsenbein eingebettet ist und die Basilarmembran enthält. Schallschwingungen erregen eine längs dieser Membran entlang fortschreitende Welle. Durch diese Auslenkung der Basilarmembran werden die Haarzellen angesprochen, die in einem geometrischen Muster längs der Basilarmembran angeordnet sind und das Muster der Anregung an das Gehirn übertragen. \grqq{} \cite[Seite 71f.]{Bernstein.2019}
\subsection{Olfaktorische Wahrnehmung}
Unser Körper ist in der Lage Gerüche schon bei sehr geringen Konzentrationen zu unterscheiden und Unterschiedlichkeiten wahrzunehmen. In der menschlichen Nase befinden sich Riechzellen in der Riechschleimhaut, die durch chemische Reize in den Rezeptoren spezifische Entladungsmuster an evolutionär ältere Gehirnareale und das limbische System senden. Das limbische System ist maßgeblich für die Emotionen des Menschen verantwortlich. \cite[Vgl. Seite 102]{Schonhammer.2013}\\
Gerüche werden oft nach dem Stoff benannt mit dem sie assoziiert werden, zum Beispiel ein blumiger Geruch bei Gerüchen, die dem Geruch von Blumen ähneln. Diese Klassifikation bei Gerüchen ist dabei nicht eindeutig, da die Zuordnung von Gerüchen mit Objekten Überschneidungen mit anderen Gerüchen haben können. Eine bessere Differenzierung von Gerüchen in Kategorien ist die Bewertung von Gerüchen nach angenehm und unangenehm. Da die olfaktorische Wahrnehmung und Gefühle eng miteinander verbunden sind. Angenehme Gerüche verursachen eine positive Stimmung und eine anziehende Gestik. \cite[Vgl. Seite 105f]{Schonhammer.2013}\\
\glqq Konzepte zur Beduftung der Innenräume von Automobilen werden u.a. unter dem Aspekt des allgemeinen Erregungszustandes des Fahrers vertreten. Gemeinsam mit Beleuchtung und Beschallung sollen Düfte den Menschen am Steuer etwa stimulieren oder beruhigen.\grqq{} \cite[Seite 122f]{Schonhammer.2013}\\
\glqq Beduftung mag als verlockende Strategie emotional wirksamer Gestaltung erscheinen, ist jedoch nicht nur aus ethischen Gründen, sondern in Rücksicht auf das Wohlbefinden unfreiwillig Betroffener problematisch. [...] Schließlich ist daran zu erinnern, dass wegen der innigen Verbindung von Gefühl und Geruch Momente der visuellen, akustischen und taktil-haptischen Gestaltung indirekt auch auf das Riechen wirken. \grqq{} \cite[Seite 123]{Schonhammer.2013}
\section{Technologien}
In den folgenden Unterbereichen werden jeweils die einzelnen Technologien, die in dieser Arbeit behandelt werden, auf Funktionsweise, Beschaffenheit und Aufbau vorgestellt. Daneben werden die wichtigsten Kenngrößen zur Beurteilung der Technologien erläutert.
\subsection{Lumineszenzdiode}
Lumineszenzdioden (LED\nomenclature{LED}{Lumineszenzdiode}) sind lichtemittierende Dioden, die Strahlen im sichtbaren oder infraroten Spektralbereich erzeugen. Dioden sind die einfachste Form von elektronischen Bauteilen und bestehen aus dotierten Halbleitermaterial mit einer pn-Schicht. Das Halbleiter-Grundmaterial bestimmt den abgestrahlten Spektralbereich des Lichtes. Liegt eine Spannung in Durchlassrichtung der Dioden an, strahlt diese in ihrem Frequenzbereich Photonen ab. \cite[Vgl. Seite 193 f.]{LofflerMang.2020} \\
Ein einfaches Beispiel soll den Zusammenhang zwischen Material und Farbe des emittierten Lichts verdeutlichen. Für die Energie von Photonen gilt nach der Einstein\grq schen Gleichung:
\begin{equation}
	E = h \cdot f = \frac{h \cdot c}{\lambda}
\end{equation}
Die Energie eines Photons ist das Produkt des Plank\grq schen Wirkungsquantums $ h = 6,626 \cdot 10^{-34}\, \mathrm{Js} $ und der Lichtgeschwindigkeit $ c = 299792458\, \frac{\mathrm{m}}{\mathrm{s}}$ durch die Wellenlänge $ \lambda $ des Photons.
Durch Umstellung folgt die Wellenlänge des Photons:
\begin{equation}
	\lambda = \frac{c \cdot h}{E}
\end{equation}
Galliumphosphid hat zum Beispiel eine Bandlücke von $ 2,25\,\mathrm{eV} $, wodurch Licht mit der Wellenlänge von
\begin{equation}
	\lambda = \frac{c \cdot h}{E} = \frac{6,626 \cdot 10^{-34}\,\mathrm{Js} \cdot 299792458\,\frac{\mathrm{m}}{\mathrm{s}}}{2,25 \cdot 1,602 \cdot 10^{-19}\, \mathrm{J}} = 551\,\mathrm{nm}
\end{equation}
im blau-grünen Spektrum emittiert wird.
Die meisten LEDs sind SMD-Bauteile (Surface-mounted device\nomenclature{SMD}{Surface-mounted device}) und sitzen in einem Kunststoff-, Keramik- oder Epoxidharzgehäuse. 
Um eine Hintergrundbeleuchtung mit Hilfe von LEDs zu erzeugen kann ein Leuchtkörper mit einer Vielzahl an LEDs hinter einer Streulichtscheibe verbaut werden, wodurch ein gleichmäßiges Licht entsteht. \cite[Vgl. Seite 194]{LofflerMang.2020} \\
Zum Erzeugen von weißem Licht strahlen drei verschiedene LEDs mit den Farben rot, grün und blau gleich hell und erst im Auge entsteht durch die Kombination ein weißes Licht. Da hier aber drei LEDs genutzt werden, ist diese Variante teurer. Der Vorteil ist, dass bei variabler Einstellung der Helligkeit der einzelnen Dioden unterschiedliche Farben für den Betrachter angezeigt werden können. \\ 
Günstiger sind Weißlicht-LEDs (WLED\nomenclature{WLED}{Weißlicht-LED}), bei denen in der Produktion auf Basis von blauen LEDs noch ein fluoreszierender Konverterstoff beigemischt wird. Dieser Stoff wird durch das blaue Licht angeregt und strahlt einen breiten Spektralbereich wieder aus, wodurch ein weißes Licht aus Primär- und Sekundärlicht entsteht. Bei WLEDs ist die Farbe nicht variabel. \cite[Vgl. Seite 194]{LofflerMang.2020} \\
Organische LED (OLED\nomenclature{OLED}{Organische LED}) besitzen einen veränderten Schichtaufbau, bei dem zwischen p- und n-Schicht eine organische Schicht aufgebracht ist. OLEDs sind dünner als normale LEDs und dadurch leichter und flexibel, wodurch sich neue Einsatzbereiche ergeben. Daneben besitzen sie eine hohe Helligkeit bei starkem Kontrast. \cite[Vgl. Seite 195]{LofflerMang.2020}
\subsection{LED-Matrix}
Eine LED-Matrix ist eine bestimmte Anordnung von LEDs in zwei orthogonalen Richtungen auf einer Ebene. Somit entsteht ein zwei dimensionales Bild.\\ 
Die Pixeldichte beschreibt, wie viele einzelne LEDs auf einer bestimmten Fläche sind. Häufig wird die Größe \glqq pixels per inch\grqq{} (ppi\nomenclature{ppi}{parts per inch}) herangezogen. Je näher der Betrachter an der Anzeigefläche steht, desto höher muss die Pixeldichte sein, damit der Betrachter einzelne Pixel nicht erkennt.
\subsection{Bildschirmtechnologien}
Unter den Bildschirmtechnologien werden folgend zwei unterschiedliche Realisierungen vorgestellt. Die erste Technologie sind aktive OLED-Displays und die Zweite passive Flüssigkristallanzeigen (LCD\nomenclature{LCD}{Flüssigkristallanzeigen}). \\
Durch die schnellen Entwicklungen bei Bildschirmtechnologien ist es nicht möglich, alle unterschiedlichen Techniken vorzustellen. Die folgenden Absätze sollen ein Grundverständnis für die möglichen Funktionsweisen liefern.\\
Aktiv bedeutet in diesem Fall, dass die Pixel das Licht selbst erzeugen, während passive Displays auf ein Hintergrundlicht angewiesen sind, da sie nur Licht abdunkeln oder durchlassen können. \\
\glqq OLED-Displays bestehen aus einem zweidimensionalen Array weißes Licht abstrahlender OLEDs, denen Farbfilter (RGB) vorgelagert sind.\grqq \cite[Seite 347]{LofflerMang.2020} Eine weitere Möglichkeit wären OLEDs mit unterschiedlichen Grundfarben (rot, grün und blau), die zusammen ein Pixel erzeugen. \\
Flüssigkristallanzeigen besitzen einen mehrschichtigen Aufbau. Die Zentrale Schicht ist eine Flüssigkristallschicht, die bei Anlegen einer Spannung an den Elektrodenschichten der Flüssigkristalle ausrichtet und die Polarisierung des einfallenden Lichtes in eine bestimmte Richtung lenkt, sodass das Licht bei einem nach geführten Polarisationsfilter entweder absorbiert oder transmittiert wird. \cite[][Vgl. Seite 346 f.]{LofflerMang.2020} Je nach Ausführung kann das Licht bei angelegter Spannung oder spannungslos transmittieren. Die Richtung und Technik der Beleuchtung der LCD variiert je nach Technik.
\subsection{Videoprojektoren}
Videoprojektoren können auf Basis unterschiedlicher Technologien für die Situation angepasst eingesetzt werden. Unterschiedliche Arten von Projektoren sind zum Beispiel LCD-, DLP- (Digital Light Processing\nomenclature{DLP}{Digital Light Processing}), LED-, LCoS- und Laser-Projektoren. \\
Zu Unterscheidung von Projektionsverfahren können diese wieder in aktive und passive Systeme eingeteilt werden. Heutzutage werden vorwiegend passive Systeme, sogenannte Lichtventilprojektoren, eingesetzt. \cite[Vgl. Seite 551]{Schmidt.2021} \\
Für die Auswahl des richtigen Projektors ist die Einsatzumgebung von Bedeutung. Je nach Helligkeit des Raumes ist die Helligkeit und der Kontrast unterschiedlich auszuwählen. Bei hoher Umgebungshelligkeit ist ein Projektor mit hoher Helligkeit vorzuziehen. Dabei ist ein niedrigerer Kontrast durch die Aufhellung der dunklen Bildbereiche durch das Umgebungslicht nicht negativ. \cite[Vgl. Seite 562]{Schmidt.2021}
\subsection{Elektronisches Papier}
Als Elektronisches Papier (E-Papier\nomenclature{E-Papier}{Elektronisches Papier}) bezeichnet man Bildschirme deren visuelle Anmutung Papier entspricht. Häufig sind diese Displays passiv, also reflektieren nur Licht und erzeugen keines. Bei manchen E-Papieren ist seitlich eine Hintergrundbeleuchtung, die über eine Folie das Display beleuchtet. Im folgenden wird die häufig verwendete Technologie der Elektrophorese für die Displays erläutert. \\
\glqq Elektronisches Papier lässt sich vereinfacht als dünne, flexible Folie beschreiben, in der in Flüssigkeit eingelagerte, elektrisch geladene Partikel (als elektronische Tinte bezeichnet) ein schwarz-weißes oder allgemein zweifarbiges Bild ergeben. Dies wird ermöglicht, indem über Elektroden elektrische Felder auf die Partikel wirken, die sich entsprechend der Ladung des angelegten Feldes ausrichten. \grqq \cite[Seite 568]{Schryen.2002} \\
Pro Pixel eines Bildes wird eine Mikrokapsel genutzt in der sich mehrere positiv geladene weiße Partikel und negativ geladene schwarze Partikel befinden. Auf beiden Seiten der Folie befinden sich Elektroden, wovon eine auf der Betrachtungsseite transparent ist. Wird auf der transparenten Elektrode eine negative Spannung und auf der inneren Elektrode eine positive, richten sich die Mikrokapseln dementsprechend aus, dass die positiven Partikel nach außen zeigen. Das Bild ist dementsprechend weiß. \cite[Vgl. Seite 567 f.]{Schryen.2002} \\
E-Papier benötigt nur beim ändern der Polarität der Pixel elektrische Energie, wodurch der Strombedarf bei seltene Bildschirmänderungen gering ist.
Vorteile gegenüber LCD-Displays sind die niedrigeren Herstellungskosten, geringeres Gewicht und die bessere Lesbarkeit. \cite[Vgl. Seite 569]{Schryen.2002}
Da bei E-Papieren vorwiegend Bilder dargestellt werden, soll im folgenden die Berechnung der Dateigröße eines Bildes für ein mögliches E-Papier erläutert werden.
Als Beispiel dient ein Schwarz-Weiß E-Papier mit einer Farbtiefe von 16 Stufen. Die Farbtiefe $ FT $\nomenclature{FT}{Farbtiefe} beträgt bei 16 Stufen zwar nur 5 Bit, um aber auf gängige Datengrößen zu kommen, sollte für die Berechnung 8 Bit pro Pixel und damit 1 Byte angenommen werden. Das Dateiformat JPEG \nomenclature{JPEG}{Joint Photographic Experts Group} nutzt 8 Bit Pro Farbkanal. Die Pixelanzahl in der Breite $ PB $ \nomenclature{PB}{Pixelanzahl in der Breite}beträgt 1920 Pixel und in die Höhe $ PH $\nomenclature{PH}{Pixelanzahl in der Höhr} 1080 Pixel. Der Komprimierungsfaktor $ KF $\nomenclature{KF}{Komprimierungsfaktor} ist eine Zahl, die angibt wie stark die reine Datenmenge des Bildes noch durch digitale Komprimierungsverfahren reduziert werden kann. Wir benutzen hier einen Komprimierungsfaktor von 12. \cite[Vgl. Seite 22]{Buhler.2018}
\begin{align}
		Datenmenge &= \frac{PB \times PH \times FT}{KF} \label{eq:Bilddatenmenge}\\
		&= \frac{1920\,\mathrm{Pixel}\times 1080\,\mathrm{Pixel} \times 8\,\frac{\mathrm{Bit}}{\mathrm{Pixel}}}{12} \cdot \frac{1}{\mathrm{Pixel}} \\
		&= 1382400\,\mathrm{Bit} = 1,3824\,\mathrm{MBit} = 172,8\,\mathrm{KByte}
\end{align}