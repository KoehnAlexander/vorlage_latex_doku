\chapter{Grundlagen}
\label{cha:Grundlagen}
Im folgenden werden für diese Arbeit notwendige Grundlagen zu unterschiedlichen Technologien und Wissensbereiche erarbeitet. Zuerst wird die menschliche Wahrnehmung erläutert, da diese die Grundlage für eine Evaluierung von Komponenten bietet, wenn diese Komponenten menschliche Sinne ansprechen. Anschließend werden die Technologien der Komponenten des Prototypen vorgestellt, um einen Einblick auf die Funktionsweise und wichtigen Kriterien zu liefern. Abschließend erfolgt eine Betrachtung der wesentlichen Wissensbereiche der Fahrzeugtechnik für die spätere Analyse.
\section{Menschliche Wahrnehmung}
Menschliche Wahrnehmung ist die "Tätigkeit oder Vorgang der Informationsaufnahme durch unsere Sinne" \cite[Seite 12]{Buhler.2017}. Dieser Prozess beschränkt sich dabei nicht nur auf die Aufnahme von Informationen, sondern auch auf die Auswahl und Bewertung der Informationsdaten nach Relevanz. \cite[Vgl. Seite 12]{Buhler.2017}\\
Mit 70 \% wahrgenommenen Umweltreize durch das Auge, ist es das bedeutendste Sinnesorgan des Menschen vor der Haut, Nase, Ohr oder Zunge. Unsere Wahrnehmung ist dabei immer eine Interpretation der erhaltenen Sinnesreize aller Sinnesorgane. Unsere visuelle Wahrnehmung ist daher nicht nur durch die Augen bestimmt, sondern auch durch die Ohren, der Nase, der Zunge und der Haut. Daneben spielen unsere Erfahrungen und emotionale Lage einen Einfluss auf die Wahrnehmung. \cite[Vgl. Seite 13 f.]{Buhler.2017}\\
Mit Medien können visuelle, auditive, haptische, motorische und olfaktorische Sinneskanäle angesprochen werden. Das Ziel der Medien ist dabei die Aufmerksamkeit des Menschen auf das Objekt zu richten. Visuelle Inhalte können Schriften, Grafiken, Animationen oder Farben sein. Auditive sind Musik oder Geräusche. Haptische Inhalte sind fühlbare Strukturen und Oberflächen, während motorische Inhalte bewegliche Teile sind. Olfaktorische Inhalte sind Düfte und Gerüche. \cite[Vgl. Seite 3]{Buhler.2017}\\
In den nächsten Unterkapiteln werden die unterschiedlichen menschlichen Wahrnehmungsarten vertieft erläutert, wobei der Fokus auf die visuelle Wahrnehmung gerichtet ist, da dort der Schwerpunkt der späteren Arbeit liegt.
%TODO Wahrnehmungspsychologie%
\subsection{Visuelle Wahrnehmung}
Die visuelle Wahrnehmung basiert hauptsächlich auf den Sinneseindrücken durch unser Auge, das lichtempfindliche Zellen besitzt und daneben wie oben erwähnt, auch aus dem Zusammenspiel der anderen Sinnesorgane.\\
Die Zellen werden dabei zwischen Stäbchen und Zapfen unterschieden. Die Mehrzahl an Zellen bilden die spektral unempfindlichen Stäbchen, mit ca. 120 Millionen, während nur eine geringe Anzahl von 7 Millionen farbempfindliche Zapfen sind. Der Unterschied in der Anzahl erklärt daher, warum das Sehen bei Dunkelheit eher schwarz weiß ist, da die Zapfen nicht genügend Licht erhalten, um ein Farbbild zu erzeugen. Dabei ist ein Zapfen immer nur für eine der drei Grundfarben rot, grün und blau lichtempfindlich. Die Farben in der menschlichen Wahrnehmung sind daher ein Ergebnis der Signalverarbeitung der drei unterschiedlichen Zapfenarten. \cite[Vgl. Seite 14]{Buhler.2017}\\
Menschen können dabei nicht von ihrem Standpunkt aus den gesamten Raum betrachten, sondern durch biologischen Gegebenheiten immer nur ein Feld, das in der Horizontalen ca. 180° Abdeckt und in der vertikalen 120°. Von diesem Blickfeld sind nur ca. 1,5° in beide Dimensionen als scharfes Bild abgedeckt. Durch Bewegungen des Auges und des Kopfes bewegt sich das scharfe Bereich und unser Gehirn fügt die scharfen Bereich zusammen. \cite[Vgl. Seite 14]{Buhler.2017}
\subsection{Haptische Wahrnehmung}
Die haptische Wahrnehmung ist ein Teilbereich der Somatosensorik. Die Sinneszellen der Somatosensorik können in drei Bereich eingeteilt werden:
\begin{itemize}
	\item Exterozeption, Wahrnehmung der Außenwelt
	\item Propriozeption, Wahrnehmung der Stellung der Gliedmaßen
	\item Interozeption, Wahrnehmung des inneren Körpers
\end{itemize}
Relevant für diese Arbeit der haptischen Wahrnehmung ist die Exterozeption beziehungsweise die Oberflächensensibilität. \cite[Vgl. Seite 26]{Sprenger.2020}\\
Die haptische Wahrnehmung erfolgt durch Rezeptoren in der Haut, die die Form, Oberfläche und Position von Objekten bestimmen. Die Rezeptoren können unterschieden werden in
\begin{itemize}
	\item Thermorezeptoren für relative Temperaturunterschiede zur Körperbefindlichkeit, 
	\item Chemorezeptoren für Stoffe,
	\item Nozirezeptoren für starke Temperaturunterschiede oder Drücke bis zur Gewebeschädigung und
	\item Mechanorezeptoren für Empfindung von Oberflächen und Druck.
\end{itemize}
\cite[Vgl. Seite 26f]{Sprenger.2020}\\
Die Verteilung der unterschiedlichen Rezeptoren ist im Körper ungleichmäßig. In der Handinnenflächen gibt es zum Beispiel Areale mit unterschiedlicher Empfindsamkeit \glqq auf Druckintensität, Geschwindigkeit einer Veränderung an der Haut oder einer Vibration. \grqq \cite[Seite 29]{Sprenger.2020}
%TODO Bei Seitenbedarf Sprenger weiter schreiben%
\subsection{Akustische Wahrnehmung}
%TODO cite Elektroakustik%
\subsection{Olfaktorische Wahrnehmung}
%TODO cite Entdecke das Riechen wieder%
Unser Körper ist in der Lage Gerüche schon bei sehr geringen Konzentrationen zu unterscheiden und unterschiedliche Konzentrationen wahrzunehmen. In der menschlichen Nase befinden sich Riechzellen in der Riechschleimhaut, die durch chemische Reize in den Rezeptoren spezifische Entladungsmuster an ältere Gehirnareale und das limbische System senden. \cite[Vgl. Seite 102]{Schonhammer.2013}\\
Gerüche werden oft nach dem Stoff benannt mit dem sie assoziiert werden, \zB blumiger Geruch bei Gerüchen, die dem Geruch von Blumen ähneln. Diese Klassifikation bei Gerüche ist dabei nicht eindeutig, da Gerüche von Objekten sich überschneiden könne. Eine bessere Differenzierung von Gerüchen in Kategorien ist die Bewertung von Gerüchen nach angenehm und unangenehm. Da die olfaktorische Wahrnehmung und Gefühle eng miteinander verbunden sind. Angenehme Gerüche verursachen eine positive Stimmung und eine anziehende Gestik. \cite[Vgl. Seite 105f]{Schonhammer.2013}\\
\glqq Konzepte zur Beduftung der Innenräume von Automobilen werden u.a. unter dem Aspekt des allgemeinen Erregungszustandes des Fahrers vertreten. Gemeinsam mit Beleuchtung und Beschallung sollen Düfte den Menschen am Steuer etwa stimulieren oder beruhigen\grqq. \cite[Seite 122f]{Schonhammer.2013}\\
\glqq Beduftung ma als verlockende Strategie emotional wirksamer Gestaltung erscheinen, ist jedoch nicht nur aus ethischen Gründen, sondern in Rücksicht auf das Wohlbefinden unfreiwillig Betroffener problematisch. [...] Schließlich ist daran zu erinnern, dass wegen der innigen Verbindung von Gefühl un dGeruch Momente der visuellen, akustischen und taktil-haptischen Gestaltung indirekt auch auf das Riechen wirken. \grqq\cite[Seite 123]{Schonhammer.2013}
\section{Technologien}
In den folgenden Unterbereichen werden jeweils die einzelnen Technologien, die in dieser Arbeit behandelt werden auf Funktionsweise, Beschaffenheit und Aufbau vorgestellt.
\subsection{Lumineszenzdiode (LED\nomenclature{LED}{Lumineszenzdiode})}
Halbleiter, Spannungsversorgung, Lichtstärke, Farben, Einbauvarianten, Streifen
\cite{G}
\subsection{LED Matrix}
Leuchtstärke, Pixeldichte, Dynamik, Eibau
\subsection{Bildschirmtechnologien}
unterschiedliche Technologien, Farbe, durchsichtig, Temperaturentwicklung
\subsection{Videoprojektoren}
Größe, Temperatur, Bildqualität, Pixel, Farbe, Reichweite, Randbedingungen Empfindlichkeit.
\subsection{Elektronisches Papier (E-Papier\nomenclature{E-Papier}{Elektronisches Papier})}
Besonderheit, Stromverbrauch, Dynamik,
\subsection{Morphende Oberflächen}
Konzepte, Einbau, Haptik, Dynamik, Komplexität

\section{Fahrzeugtechnik}
Für die spätere Analyse der Komponenten sind neben den oben erläuterten Technologien das Gesamtkontext der Fahrzeugtechnik wichtig, um ein Verständnis für die Herausforderungen zu entwickeln.\\
Deswegen wird im folgenden auf die allgemeinen Prozessschritte der Fahrzeugentwicklung, die Fahrzeugkarosserie, die Elektrik/ Elektronik Architektur, die rechtlichen Rahmenbedingungen und die sicherheitsrelevanten Vorkehrungen eingegangen.
\subsection{Fahrzeugentwicklung}
\subsection{Karosserie}
\subsection{Elektrik/Elektronik Architektur (E/E Architektur\nomenclature{E/E Architektur}{Elektrik/Elektronik Architektur})}
Die


\subsection{Rechtliche Rahmenbedingungen}

\subsection{Sicherheitsanforderungen}