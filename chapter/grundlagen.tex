\chapter{Grundlagen}
\label{cha:Grundlagen}
Im folgenden werden für diese Arbeit notwendige Grundlagen zu unterschiedlichen Technologien und Wissensbereiche erarbeitet. Zuerst werden die unterschiedlichen menschlichen Sinneswahrnehmungen erläutert. Anschließend werden die Technologien der Komponenten des Prototypen vorgestellt. Abschließend erfolgt eine Betrachtung der wesentlichen Wissensbereiche der Fahrzeugtechnik für die spätere Analyse.
\section{Menschliche Wahrnehmung}
Menschliche Wahrnehmung ist die "Tätigkeit oder Vorgang der Informationsaufnahme durch unsere Sinne" \cite[Seite 12]{Buhler.2017}. Dieser Prozess beschränkt sich dabei nicht nur auf die Aufnahme von Informationen, sondern auch auf die Auswahl und Bewertung der Informationsdaten nach Relevanz. \cite[Vgl. Seite 12]{Buhler.2017}\\
Mit 70 \% wahrgenommenen Umweltreize durch das Auge, ist es das bedeutendste Sinnesorgan des Menschen vor der Haut, Nase, Ohr oder Zunge. Unsere Wahrnehmung ist dabei immer eine Interpretation der erhaltenen Sinnesreize aller Sinnesorgane. Unsere visuelle Wahrnehmung ist daher nicht nur durch die Augen bestimmt, sondern auch durch die Ohren, der Nase, der Zunge und der Haut. Daneben spielen unsere Erfahrungen und emotionale Lage einen Einfluss auf die Wahrnehmung. \cite[Vgl. Seite 13 f.]{Buhler.2017}\\
Mit digitalen Medien können visuelle, auditive, haptische, motorische und olfaktorische Sinneskanäle angesprochen werden. Das Ziel der Medien ist dabei die Aufmerksamkeit des Menschen auf das Objekt zu richten. Visuelle Inhalte können Schriften, Grafiken, Animationen oder Farben sein. Auditive sind Musik oder Geräusche. Haptische Inhalte sind fühlbare Strukturen und Oberflächen, während motorische Inhalte bewegliche Teile sind. Olfaktorische Inhalte sind Düfte und Gerüche. \cite[Vgl. Seite 3]{Buhler.2017}\\
In den nächsten Unterkapiteln werden die unterschiedlichen menschlichen Wahrnehmungsarten erläutert, wobei der Fokus auf die visuelle Wahrnehmung gerichtet ist.
\subsection{Visuelle Wahrnehmung}
Die visuelle Wahrnehmung basiert hauptsächlich auf den Sinneseindrücken durch unser Auge, das lichtempfindliche Zellen besitzt und daneben wie oben erwähnt, auch aus dem Zusammenspiel der anderen Sinnesorgane.\\
Die Zellen werden dabei zwischen Stäbchen und Zapfen unterschieden. Die Mehrzahl an Zellen bilden die spektral unempfindlichen Stäbchen, während nur eine geringe Anzahl farbempfindliche Zapfen sind. Dabei ist ein Zapfen immer nur für eine der drei Grundfarben rot, grün und blau lichtempfindlich. Die Farben in der menschlichen Wahrnehmung sind daher ein Ergebnis der Signalverarbeitung der drei unterschiedlichen Zapfen. \cite[Vgl. Seite 14]{Buhler.2017}

\subsection{Haptische Wahrnehmung}
\subsection{Akustische Wahrnehmung}
\subsection{Olfaktorische Wahrnehmung}

\section{Technologien}
In den folgenden Unterbereichen werden jeweils die einzelnen Technologien auf Funktionsweise, Beschaffenheit und Aufbau vorgestellt, um diese Technologien auf ihre Anforderungen einordnen zu können.
\subsection{Light-emitting diode (LED\nomenclature{LED}{Light-emitting diode})}
Halbleiter, Spannungsversorgung, Lichtstärke, Farben, Einbauvarianten, Streifen
\subsection{LED Matrix}
Leuchtstärke, Pixeldichte, Dynamik, Eibau
\subsection{Bildschirmtechnologien}
unterschiedliche Technologien, Farbe, durchsichtig, Temperaturentwicklung
\subsection{Videoprojektoren}
Größe, Temperatur, Bildqualität, Pixel, Farbe, Reichweite, Randbedingungen Empfindlichkeit.
\subsection{Elektronisches Papier}
Besonderheit, Stromverbrauch, Dynamik,
\subsection{Morphende Oberflächen}
Konzepte, Einbau, Haptik, Dynamik, Komplexität

\section{Fahrzeugtechnik}
Für die spätere Analyse der Komponenten sind neben den oben erläuterten Technologien das Gesamtkontext der Fahrzeugtechnik wichtig, um ein Verständnis für die Herausforderungen zu entwickeln.\\
Deswegen wird im folgenden auf die allgemeinen Prozessschritte der Fahrzeugentwicklung, die Fahrzeugkarosserie, die Elektrik/ Elektronik Architektur, die rechtlichen Rahmenbedingungen und die sicherheitsrelevanten Vorkehrungen eingegangen.
\subsection{Fahrzeugentwicklung}
\subsection{Karosserie}
\subsection{Elektrik/Elektronik Architektur (E/E Architektur\nomenclature{E/E Architektur}{Elektrik/Elektronik Architektur})}
Die


\subsection{Rechtliche Rahmenbedingungen}

\subsection{Sicherheitsanforderungen}