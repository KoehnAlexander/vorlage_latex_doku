\chapter{Grundlagen}
\label{cha:Grundlagen}
Im folgenden werden für diese Arbeit notwendige Grundlagen zu unterschiedlichen Technologien und Wissensbereiche erarbeitet. Zuerst werden die Grundlagen für die unterschiedlichen menschlichen Sinneswahrnehmungen erläutert. Anschließend werden die Technologien der Komponenten des Prototypen vorgestellt. Abschließend erfolgt eine Betrachtung der wesentlichen Wissensbereiche der Fahrzeugtechnik für die spätere Analyse.
\section{Wahrnehmung des Menschen}
\subsection{Optische Wahrnehmung}
\subsection{Haptische Wahrnehmung}
\subsection{Akustische Wahrnehmung}
\subsection{Olfaktorische Wahrnehmung}

\section{Technologien}
In den folgenden Unterbereichen werden jeweils die einzelnen Technologien auf Funktionsweise, Beschaffenheit und Aufbau vorgestellt, um diese Technologien auf ihre Anforderungen einordnen zu können.
\subsection{Light-emitting diode (LED\nomenclature{LED}{Light-emitting diode})}
Halbleiter, Spannungsversorgung, Lichtstärke, Farben, Einbauvarianten, Streifen
\subsection{LED Matrix}
Leuchtstärke, Pixeldichte, Dynamik, Eibau
\subsection{Bildschirmtechnologien}
unterschiedliche Technologien, Farbe, durchsichtig, Temperaturentwicklung
\subsection{Videoprojektoren}
Größe, Temperatur, Bildqualität, Pixel, Farbe, Reichweite, Randbedingungen Empfindlichkeit.
\subsection{Elektronisches Papier}
Besonderheit, Stromverbrauch, Dynamik,
\subsection{Morphende Oberflächen}
Konzepte, Einbau, Haptik, Dynamik, Komplexität

\section{Fahrzeugtechnik}
Für die spätere Analyse der Komponenten sind neben den oben erläuterten Technologien das Gesamtkontext der Fahrzeugtechnik wichtig, um ein Verständnis für die Herausforderungen zu entwickeln.\\
Deswegen wird im folgenden auf die allgemeinen Prozessschritte der Fahrzeugentwicklung, die Fahrzeugkarosserie, die Elektrik/ Elektronik Architektur, die rechtlichen Rahmenbedingungen und die sicherheitsrelevanten Vorkehrungen eingegangen
\subsection{Fahrzeugentwicklung}
\subsection{Karosserie}
\subsection{Elektrik/Elektronik Architektur}

\subsection{Rechtliche Rahmenbedingungen}

\subsection{Sicherheitsanforderungen}