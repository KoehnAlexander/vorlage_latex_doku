\chapter{Grundlagen}
\label{cha:Grundlagen}

%... Theoretische Grundlagen (vielleicht auch zitiert aus %Standardwerken, wie z.B. aus \autocite{Tipler.2019}), %Rechercheergebnisse, Stand der Technik \index{Stand der Technik} (ggf. %zitiert aus Hochschulschriften, welche Online verfügbar sind, wie %z.B.~\autocite{Ziegler.2017}), etc.
Im folgenden werden für diese Arbeit notwendige Grundlagen erläutert. Die Reihenfolge der Erklärungen geht von allgemeinen Grundlagen zu fahrzeugspezifischen Kenntnissen.
\section{Digitale Technologien}
\subsection{Künstliche Intelligenz}
\subsection{Deep Learning}

\subsection{Blockchain}

\subsection{Non-Fungible Token}
\subsection{Web3}
\subsection{Virtual Reality}
\subsection{Augmented Reality}
\section{Technologien}
\subsection{LED}
\subsection{Matrix LED}
\subsection{Displays}
\subsection{Projektoren}
\subsection{E Ink Folien}
\subsection{Morphende Oberflächen}
\section{Fahrzeugtechnik}
\subsection{Fahrzeugentwicklung}
\subsection{Elektrik/Elektronik Architektur}
\subsection{Bussysteme}
\subsection{Rechtliche Rahmenbedingungen}
\subsection{Sicherheitsbedingungen}

