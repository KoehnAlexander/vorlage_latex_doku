\addchap{Anhang}
\setcounter{chapter}{1}
\section{Einheiten von Bit und Byte}
\begin{table}[hbt]	
	\centering
	\renewcommand{\arraystretch}{1.5}	% Skaliert die Zeilenhöhe der Tabelle
	\captionabove[Liste der Einheiten von Bit und Byte]{Liste der Einheiten von Bit und Byte}
	\label{tab:EinheitenBitByte}
	\begin{tabular}{c|cccc}
		\textbf{Einheit} & \parbox[t]{0.17\linewidth}{\centering Anzahl an Bit} & \parbox[t]{0.17\linewidth}{\centering Anzahl an Byte }  & \parbox[t]{0.17\linewidth}{\centering Anzahl an Kilobit} & \parbox[t]{0.17\linewidth}{\centering Anzahl an Kilobyte}\\ 
		\hline 
		\hline
		1 Bit & 1 & - & - & - \\
		1 Byte & 8 & 1 & - & - \\
		1 kBit & 1.000 & 125 & 1 & - \\
		1 kByte & 8.000 & 1.000 & 8 & 1 \\
		1 MBit & 1.000.000 & 125.000 & 1.000 & 125 \\
		1 MByte & 8.000.000 & 1.000.000 & 8.000 & 1.000 \\
		1 Gbit & 1.000.000.000 & 125.000.000 & 1.000.000 & 125.000 \\
		1 GByte & 8.000.000.000 & 1.000.000.000 & 8.000.000 & 1.000.000 \\
	\end{tabular} 
\end{table}