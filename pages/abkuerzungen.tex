% alle Abkürzungen, die in der Arbeit verwendet werden. Die Alphabetische Sortierung übernimmt Latex. Nachfolgend sind Beispiele genannt, welche nach Bedarf angepasst, gelöscht oder ergänzt werden können.

% Bei den unten stehenden Formelzeichen ist erläutert, wie explizite Sortierschlüssel über den Inhalt der eckigen Klammer angegeben werden.

% Allgemeine Abkürzungen %%%%%%%%%%%%%%%%%%%%%%%%%%%%
%\nomenclature{Abb.}{Abbildung}
%\nomenclature{bzw.}{beziehungsweise}
%\nomenclature{DHBW}{Duale Hochschule Baden-Württemberg}
%\nomenclature{ebd.}{ebenda}
%\nomenclature{et al.}{at alii}
%\nomenclature{etc.}{et cetera}
%\nomenclature{evtl.}{eventuell}
\nomenclature{f.}{folgende Seite}
\nomenclature{ff.}{fortfolgende Seiten}
%\nomenclature{ggf.}{gegebenenfalls}
%\nomenclature{Hrsg.}{Herausgeber}
%\nomenclature{Tab.}{Tabelle}
%\nomenclature{u. a.}{unter anderem}
%\nomenclature{usw.}{und so weiter}
\nomenclature{vgl.}{vergleiche}
%\nomenclature{z. B.}{zum Beispiel}
%\nomenclature{z. T.}{zum Teil}
\nomenclature{ca.}{circa}

% Dateiendungen %%%%%%%%%%%%%%%%%%%%%%%%%%%%%%%%%%%%
\nomenclature{EMF}{Enhanced Metafile}
%\nomenclature{JPG}{Joint Photographic Experts Group}
\nomenclature{PDF}{Portable Document Format}
\nomenclature{PNG}{Portable Network Graphics}
%\nomenclature{XML}{Extensible Markup Language}


% Formelzeichen %%%%%%%%%%%%%%%%%%%%%%%%%%%%%%%%%%%%
%\nomenclature[a]{$a$}{Beschleunigung}
%\nomenclature[F]{$F$}{Kraft}
\nomenclature[m]{$m$}{Masse}
%\nomenclature[P]{$P$}{Leistung}
\nomenclature[U]{$U$}{Spannung}
\nomenclature[R]{$R$}{Widerstand}
\nomenclature[P]{$P$}{Leistung}

% Einheiten %%%%%%%%%%%%%%%%%%%%%%%%%%%%%%%%%%%%%%%%
%\nomenclature{m}{Meter}
\nomenclature{cm}{Zentimeter}
\nomenclature{mm}{Millimeter}



