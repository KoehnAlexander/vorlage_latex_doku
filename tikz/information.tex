\begin{tikzpicture}[first style/.style={rectangle, rounded corners, align=center}, steuer style/.style={draw, rectangle, text width=2.8cm, align=center}]
	\shade[left color=cyan!30!white, right color=blue!40!white] (1, 5.5) rectangle (8,-3);
	\shade[left color=green!30!white, right color=cyan!30!white] (-5, 5.5) rectangle (1,-3);
	\draw (-3, 3) arc (-170:170:12mm);
	\draw[very thick] (-2.4, 2.25) arc (-120:120:11mm) ;
	\draw[->, red] (-5, 3.2) -- node[above] {Licht} (-3.5 , 3.2);
	\node[first style] (A) at (-2, 4.6) {Auge};
	\node[first style, ultra thick] (P1) at (-4, 5){\textbf{Physik}};
	\node[first style, ultra thick] (P2) at (1, 5){\textbf{Physiologie}};
	\node[first style, right=of P2, ultra thick] (P3) at (4, 5) {\textbf{Psychologie}};
	\node[steuer style] (S6) at (5, 3.5) {Nervöse Filtersteuerung};
	\node[steuer style, thick] (S1) at (5, 0) {Wahrnehmung};
	\node[steuer style] (S5) at (1, 1.5) {Nervöse Informationsverarbeitung};
	\node[steuer style] (S2) at (5, -2)  {Stimmungen und Gefühle};
	\node[steuer style] (S4) at (1, 0) {Andere Sinnesorgane};
	\node[steuer style] (S3)  at (1, -2) {Gespeicherte Erfahrung};
	\draw[very thick] (S5) -- (-0.8, 2.9);
	\draw (S1) -- (S2);
	\draw (S1) -- (S3);
	\draw (S1) -- (S4);
	\draw (S1) -- (S5);
	\draw (S1) -- (S6);
	\draw (S5) -- (S6);
	\draw (S2) -- (S3);
	\draw (S4) -- (S3);
\end{tikzpicture}